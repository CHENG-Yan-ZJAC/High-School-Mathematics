\section{本章小结}

本章介绍数列,特别介绍了两个数列:等差数列和等比数列。本章有两个难点,$a_n$一般不会直接是等差或等比,而是加上一个偏移量或倒数后构成等差或等比,其次要注意等差和等比的平均数。

%============================================================
\subsection{习题}

\begin{example}[综合运用12,难度:$\star \star $]
已知数列$\left\{ a_n \right\} $的前$n$项的和为$S_n$,且$a_{n+1}=2S_n+2,n\in \mathbb{N} ^+$。
\begin{enumerate}
    \item 求数列$\left\{ a_n \right\} $的通项公式。
    \item 在$a_n$与$a_{n+1}$之间插入$n$个数,使这$n+2$个数组成一个公差为$d_n$的等差数列,在数列$\left\{ d_n \right\} $中是否存在3项$d_m,d_k,d_p$(其中	$m,k,p$成等差数列)成等比数列?若存在,求出这样的3项;若不存在,请说明理由。
\end{enumerate}
\end{example}

解:

(1)先分析表达式:
\begin{align*}
&\because a_{n+1}=2S_n+2 \quad a_n=2S_{n-1}+2 \\
&\therefore a_{n+1}-a_n=2a_n \\
&\therefore a_n=a_1\cdot 3^{n-1}
\end{align*}
可见是等比数列,再求首项:
\begin{align*}
&\because a_{n+1}=2S_n+2 \\
&\therefore a_2=2a_1+2 \\
&\therefore 3a_1=2a_1+2 \\
&\therefore a_1=2
\end{align*}
最终得到$a_n=2\cdot 3^{n-1}$。

(2)
\begin{align*}
&\because a_n=2\cdot 3^{n-1},a_{n+1}=2\cdot 3^n \\
&\therefore d_n=\frac{2\cdot 3^n-2\cdot 3^{n-1}}{n+1}=\frac{4\cdot 3^{n-1}}{n+1} \\
&\because d_m\cdot d_p={d_k}^2 \\
&\therefore \frac{4\cdot 3^{m-1}}{m+1}\cdot \frac{4\cdot 3^{p-1}}{p+1}=\frac{4\cdot 3^{k-1}}{k+1}\frac{4\cdot 3^{k-1}}{k+1} \\
&\because m+p=2k \\
&\therefore \frac{3^{2k-2}}{mn+2k+1}=\frac{3^{2k-2}}{k^2+2k+1} \\
&\therefore mn=k^2
\end{align*}
得到$m+p=2k$且$mp=k^2$,联立两式得:
\[
\left( m-p \right) ^2=0
\]
不存在。

\begin{tcolorbox}
本题难度不大,但较为复杂。
\end{tcolorbox}




