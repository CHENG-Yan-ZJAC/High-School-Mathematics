\section{等差数列}

本节要点:
\begin{itemize}
    \item 掌握等差数列的概念;
    \item 掌握等差数列的通项公式和求和公式。
\end{itemize}

%============================================================
\subsection{等差数列的概念}

\[
a_n=a_1+\left( n-1 \right) d \qquad n=1,2,3,\cdots
\]

需要注意:
\begin{itemize}
    \item 确定一个等差数列需要解决两个未知量,所以有两个条件就能求解一个等差数列;
    \item 没有说$d>0$,事实上,$d$可以是任意实数;
    \item 等差数列有$a_n=\left( a_{n-1}+a_{n+1} \right) /2$,对应了算术平均;
    \item 等差数列可以视作直线$f\left( x \right) =kx+c$的离散化,$d$可视为“斜率”。
\end{itemize}

\begin{theorem}
设两个等差数列$\left\{ a_n \right\} ,\left\{ b_n \right\} $,则$\left\{ pa_n+qb_n \right\} $也是等差数列。
\end{theorem}

%============================================================
\subsection{等差数列的前\texorpdfstring{$n$}{n}项和公式}

\[
S_n=\frac{n\left( a_1+a_n \right)}{2}=na_1+\frac{n\left( n-1 \right)}{2}d \qquad n=1,2,3,\cdots
\]

\begin{tcolorbox}
反复体会求和公式的推导过程。
\end{tcolorbox}

或许写成如下方式更能发现问题:
\begin{align*}
&S_n=\frac{d}{2}n^2+\left( a_1-\frac{d}{2} \right) n \\
&\frac{S_n}{n}=\frac{d}{2}n+\left( a_1-\frac{d}{2} \right)
\end{align*}
可见$S_n$是一条二次曲线,其次$\frac{S_n}{n}$是一个等差数列。如果写成如下形式:
\[
a_n=\frac{S_n-S_{n-1}}{n-\left( n-1 \right)}
\]
体会“$a_n$是$S_n$在$n$点的斜率”这个说法,这就是差分方程的雏形。

%============================================================
\subsection{习题}

\begin{example}[综合运用8,难度:$\star $]
已知两个等差数列$2,6,10,\cdots ,190$及$2,8,14,\cdots ,200$,将这两个等差数列的公共项按从小到大的顺序组成一个新数列。
求这个新数列的各项之和。
\end{example}

解:

易得:
\begin{align*}
&a_n=2+4\left( n-1 \right) \qquad n=1,2,\cdots ,48 \\
&b_m=2+6\left( m-1 \right) \qquad m=1,2,\cdots ,34
\end{align*}
求解公共项:
\begin{align*}
&\because 2+4\left( n-1 \right) =2+6\left( m-1 \right) \\
&\therefore 2n+1=3m
\end{align*}
可见$3m$是奇数,考虑到$2n+1$最大取到97,于是$m$取值:
\[
m=1,3,5,7,9,11,13,15,17,19,21,23,25,27,29,31
\]
新数列为首项2公差12的等差数列,共16项,于是:
\[
16\cdot 2+\frac{16\cdot \left( 16-1 \right)}{2}\cdot 12=1472
\]

\begin{tcolorbox}
本题考察项的表达式。
\end{tcolorbox}




