\section{数列的概念}

本节要点:
\begin{itemize}
    \item 掌握数列的概念;
    \item 掌握数列前$n$项和的概念。
\end{itemize}

~

数列及其前$n$项的和没有难度。反复体会本节第2个思考题,如何从$S_n$推导$a_n$,具有典型意义。

注意两点:
\begin{itemize}
    \item 数列可以看作是函数的离散化,所以所有对函数的考察和工具,都应用于数列;
    \item 在微积分中,我们一般不关心$a_n$,而是关心当$n\rightarrow \infty $时$S_n$的敛散性,称为级数,因为级数可以描述一个数。
\end{itemize}

%============================================================
\subsection{习题}


\begin{example}[综合运用5,难度:$\star $]
传说古希腊毕达哥拉斯学派的数学家用沙砾和小石子来研究数。他们根据沙砾或小石子所排列的形状把数分成许多类,如图中第一行的1,3,6,10称为三角形数,第二行的1,4,9,16称为正方形数,第三行的1,5,12,22称为五边形数。请你分别写出三角形数、正方形数、五边形数所构成的数列的第5项和第6项。
\end{example}

解:

三角形数。观察图形发现规律:
\[
a_n-a_{n-1}=n
\]
也就是说它们的差构成公差为1的等差数列。于是,可另$b_n=n$,得:
\[
a_n=S_n=\sum_{i=1}^n{b_i}=\frac{n\left( 1+n \right)}{2}
\]

正方形数。不难发现是正方形的面积,于是:
\[
a_n=n^2
\]

五边形数。令$\varDelta a_n=a_n-a_{n-1}$,不难发现
\[
\varDelta a_n=3\left( n-1 \right) +1
\]
也是一个等比数列,于是
\[
a_n=S_n=\sum_{i=1}^n{\varDelta a_n}=\frac{n\left[ 1+3\left( n-1 \right) +1 \right]}{2}=\frac{n\left( 3n-1 \right)}{2}
\]

\begin{tcolorbox}
本题只是求解具体的项,没有难度。若是求通项公式,需要仔细观察。
\end{tcolorbox}

~

\begin{example}[拓广探索7,难度:$\star $]
已知函数$f\left( x \right) =\frac{2^x-1}{2^x},x\in \mathbb{R} $,设数列$\left\{ a_n \right\} $的通项公式为$a_n=f\left( n \right) ,n\in \mathbb{N} ^+$。
\begin{enumerate}
    \item 求证$a_n\geqslant \frac{1}{2}$。
    \item $\left\{ a_n \right\} $是递增数列还是递减数列?为什么?
\end{enumerate}
\end{example}

解:

(1、2)易得$a_n=\frac{2^n-1}{2^n}=1-\left( \frac{1}{2} \right) ^n$,可令$n_1<n_2$
\[
a_{n_1}-a_{n_2}=\left[ 1-\left( \frac{1}{2} \right) ^{n_1} \right] -\left[ 1-\left( \frac{1}{2} \right) ^{n_2} \right] =\left( \frac{1}{2} \right) ^{n_2}-\left( \frac{1}{2} \right) ^{n_1}<0
\]
可见$a_n\geqslant \frac{1}{2}$递增,于是
\[
a_n\geqslant a_1=\frac{1}{2}
\]

\begin{tcolorbox}
本题没有难度,但本题是一个很好的数列和函数关系的例子。
\end{tcolorbox}




