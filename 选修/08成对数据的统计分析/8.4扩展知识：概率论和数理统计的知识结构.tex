\section{扩展知识:概率论和数理统计的知识结构}

概率论和数理统计这门学科,国外分为两本教材,分别是“概率论”和“数理统计”,国内作为理工科入门教材一般都合在一本。学科分为两部分,概率论和数理统计,前者是后者的理论基础,后者是前者的工程应用。

概率论一般分4大部分:
\begin{itemize}
    \item 随机事件:主要介绍概率的相关知识,如条件概率、全概率、贝叶斯公式、古典概型、几何概型等;
    \item 随机变量:将随机事件抽象,用函数这一工具讨论概率,主要讲随机变量的分布,如二项分布、泊松分布、正态分布等,若干个随机变量的关系,如边缘分布、条件分布、独立性等;
    \item 数字特征:讲述数学期望、方差、协方差、相关系数、矩等;
    \item 大数定理和中心极限定理:作为概率论的补完。
\end{itemize}

数理统计一般分3大部分:
\begin{itemize}
    \item 基础:主要介绍基本概念,如总体、样本、抽样;
    \item 参数估计:讲述如何用样本估计总体的参数,如估计总体的期望、方差等,以及对估计方法的评价;
    \item 假设检验:对总体的一些假设,我们如何用已知的样本进行对这个假设的判断,或者说我们如何用数学的方法定量地判断一个假设的可信度。
\end{itemize}

可见,高中选修教材第三册中,除了本章第三节是属于数理统计的假设检验部分,其他都是概率论部分的内容。概率论和数理统计虽然视作数学,但其实是建立在微积分和线性代数的基础上。若缺乏这两个基础,所有知识点都不可能进行有效深入地讨论。




