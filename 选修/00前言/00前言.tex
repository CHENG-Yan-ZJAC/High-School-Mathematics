\chapter{前言}

数学选修,上中下三册,讲了3大部分内容:
\begin{itemize}
    \item 几何:用代数的方法讨论立体几何和平面几何;
    \item 代数:介绍导数的概念;
    \item 概率论:简单介绍概率论的基础知识。
\end{itemize}

~

章节编排:
\begin{itemize}
    \item {\bf 第一章\ 空间向量与立体几何},49页:用代数的方法讨论几何,立体几何到此介绍完毕;
    \item {\bf 第二章\ 直线和圆的方程},54页:用代数的方法讨论平面的直线和圆;
    \item {\bf 第三章\ 圆锥曲线的方程},43页:用代数的方法讨论平面的椭圆、双曲线、抛物线,平面几何到此介绍完毕;
    \item {\bf 第四章\ 数列},57页:介绍等差数列和等比数列;
    \item {\bf 第五章\ 一元函数的导数及其应用},47页:简单介绍微积分的基础——导数,至此代数介绍完毕;
    \item {\bf 第六章\ 计数原理},38页:主要讲排列组合,为下一章铺垫;
    \item {\bf 第七章\ 随机变量及其分布},49页:介绍概率论的基础知识;
    \item {\bf 第八章\ 成对数据的统计分析},50页:继续介绍概率论的基础知识和数理统计中的假设检验,至此概率论介绍完毕。
\end{itemize}

~

本笔记选取有价值的课后习题,给出求解和难度评价。

难度评价标准:
\begin{itemize}
    \item $\star $:简单,考察基本定义等概念,解题方法有章可循;
    \item $\star \star $:中等,考察“定理—定义—性质”的数学范式,解题方法依然有章可循,可能多绕几个弯,类似“定理—定义1—性质—定义2—性质……”;
    \item $\star \star \star $:较难,需融汇数个知识点,要求对这几个知识点在方法论层面融会贯通。
    \item $\star \star \star \star $:最难,考察实际问题,并没有直接给出明确的数学问题,需自行建模,所以需要我们在方法论和哲学层面融会贯通所有知识点。
\end{itemize}




