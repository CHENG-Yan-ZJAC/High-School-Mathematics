\section{导数的概念及其意义}

本节要点:
\begin{itemize}
    \item 掌握导数的概念;
    \item 理解导数的几何意义。
\end{itemize}

%============================================================
\subsection{变化率的问题}

\begin{definition}[增量和变化率]
我们定义,{\bf 自变量的增量}是两个自变量的差,{\bf 因变量的增量}是两个因变量的差,即:
\begin{align*}
&\Delta x:=x_1-x_0 \\
&\Delta y:=y_1-y_0=f\left( x_1 \right) -f\left( x_0 \right)
\end{align*}
我们继续定义两个增量的比值为{\bf 函数的变化率},即:
\[
\frac{\Delta y}{\Delta x}=\frac{f\left( x_1 \right) -f\left( x_0 \right)}{x_1-x_0}=\frac{f\left( x_0+\Delta x \right) -f\left( x_0 \right)}{\Delta x}
\]
\end{definition}

显然,函数的变化率是一个和自变量及其增量区间有关的新函数。当取相同的$\Delta x$时,变化率描述了函数的变化快慢。变化率越大,说明函数在同等$\Delta x$下的变化越大。

%============================================================
\subsection{导数的概念及其几何意义}

下面我们考察函数在一个点上的变化率,即当$x_1\rightarrow x_0$(或$\Delta x\rightarrow 0$),函数的变化率的存在性和取值。

\begin{definition}[导数]
设函数$f\left( x \right) $在点$x_0$的某邻域$N\left( x_0 \right) $内有定义,若当$\Delta x\rightarrow 0$时$\Delta y/\Delta x$的极限存在,则称{\bf 函数$f\left( x \right) $在$x_0$处可导},并称此极限值为{\bf 函数$f\left( x \right) $在点$x_0$的导数(derivative)},记为$f'\left( x_0 \right) $,即:
\[
f'\left( x_0 \right) :=\underset{\Delta x\rightarrow 0}{\lim}\frac{\Delta y}{\Delta x} \quad \text{或} \quad f'\left( x_0 \right) :=\underset{x_1\rightarrow x_0}{\lim}\frac{f\left( x_1 \right) -f\left( x_0 \right)}{x_1-x_0}
\]
也可用莱布尼兹(Leibniz)记号记为:
\[
\left. y' \right|_{x=x_0} \quad \text{或} \quad \left. \frac{dy}{dx} \right|_{x=x_0}
\]
\end{definition}

从定义上来讲,函数$f\left( x \right) $在$x_0$处的导数是一个极限,是一个可计算的确定的数。具体还有左导数和右导数,略。

\begin{definition}[导函数]
如果函数$f\left( x \right) $在区间$D$内每一点都可导,则称每个导数构成的新函数为{\bf $f\left( x \right) $的导函数},记为$f'\left( x \right) $(或$y'$)。显然,导函数$f'\left( x \right) $在$x_0$的值就是导数$f'\left( x_0 \right) $。
\end{definition}

除非特别指明,一般我们将导函数简称为导数。区间$\left( a,b \right) $(或$\left[ a,b \right] $)内所有可导函数的集合通常记作$D\left( a,b \right) $(或$D\left[ a,b \right] $),所以$f\left( x \right) $在区间$D$内可导也可记作$f\left( x \right) \in D\left( a,b \right) $(或$f\left( x \right) \in D\left[ a,b \right] $)。

求导是对函数的一种运算。运算对象是函数,得到的结果是另一个函数。从集合角度,求导是一个函数集合到另一个函数集合的映射。

\begin{tcolorbox}
在讨论导数的时候,紧紧抓住导数的定义,从定义出发。后续关于导数的定理和公式都是用定义证明和推导。
\end{tcolorbox}

从几何上来讲,导数的意义是切线的斜率。物理上,导数描述了一个物理量的瞬时变化速度,比方说路程的导数是速度,速度的导数是加速度。导数为求解物理量的变化速度提供了强大的数学工具。




