\section{导数的运算}

本节要点:
\begin{itemize}
    \item 掌握基本初等函数的求导公式;
    \item 掌握导数的运算法则;
    \item 掌握复合函数的求导方法。
\end{itemize}

%============================================================
\subsection{基本初等函数的导数}

\begin{align*}
\begin{matrix}
	\left( C \right) '=0 \hfill                                       & \left( x^a \right) '=a\cdot x^{a-1} \hfill \\
	\left( a^x \right) '=a^x\ln a \hfill                              & \left( e^x \right) '=e^x \hfill \\
	\left( \log _ax \right) '=\frac{1}{x\ln a} \hfill                 & \left( \ln x \right) '=\frac{1}{x} \hfill \\
	\left( \sin x \right) '=\cos x \hfill                             & \left( \cos x \right) '=-\sin x \hfill \\
	\left( \tan x \right) '=\sec ^2x \hfill                           & \left( \cot x \right) '=-\csc ^2x \hfill \\
	\left( \sec x \right) '=\sec x\cdot \tan x \hfill                 & \left( \csc x \right) '=-\csc x\cdot \cot x \hfill \\
	\left( \mathrm{arc}\sin x \right) '=\frac{1}{\sqrt{1-x^2}} \hfill & \left( \mathrm{arc}\cos x \right) '=-\frac{1}{\sqrt{1-x^2}} \hfill \\
	\left( \mathrm{arc}\tan x \right) '=\frac{1}{1+x^2} \hfill        & \left( \mathrm{arc}\cot x \right) '=-\frac{1}{1+x^2} \hfill \\
	\left( \mathrm{sh}x \right) '=\mathrm{ch}x \hfill                 & \left( \mathrm{ch}x \right) '=\mathrm{sh}x \hfill \\
\end{matrix}
\end{align*}

以上基本公式中,除了$\left( C \right) ',\left( a^x \right) ',\left( \sin x \right) '$,其余都可以从这三个公式导出。

{\bf 计算$\left( C \right) '$}
\[
\left( C \right) '=\underset{\Delta x\rightarrow 0}{\lim}\frac{C-C}{\Delta x}=0
\]

{\bf 计算$\left( a^x \right) '$}
\[
\left( a^x \right) '=\underset{\Delta x\rightarrow 0}{\lim}\frac{a^{x+\Delta x}-a^x}{\Delta x}=a^x\underset{\Delta x\rightarrow 0}{\lim}\frac{a^{\Delta x}-1}{\Delta x}=a^x\ln a
\]

{\bf 计算$\left( \sin x \right) '$}
\begin{align*}
\left( \sin x \right) '&=\underset{\Delta x\rightarrow 0}{\lim}\frac{\sin \left( x+\Delta x \right) -\sin x}{\Delta x} \\
&=\underset{\Delta x\rightarrow 0}{\lim}\frac{2\sin \frac{\Delta x}{2}\cos \frac{2x+\Delta x}{2}}{\Delta x}=\underset{\Delta x\rightarrow 0}{\lim}cos \frac{2x+\Delta x}{2}=\cos x
\end{align*}

%============================================================
\subsection{导数的四则运算法则}

\begin{align*}
&\left( C\cdot f \right) '=C\cdot f' \\
&\left( f\pm g \right) '=f'\pm g' \\
&\left( f\cdot g \right) '=f'g+fg' \\
&\left( \frac{f}{g} \right) '=\frac{f'g-fg'}{g^2} \\
&\left( \frac{1}{g} \right) '=-\frac{g'}{g^2}
\end{align*}

前两个公式表明导数运算是线性的。

%============================================================
\subsection{简单的复合函数的导数}

\begin{theorem}[复合函数求导定理]
若函数$y=y\left( u \right) ,u=u\left( x \right) $在对应区间可导,则复合函数$y=y\left[ u\left( x \right) \right] $在对应区间也可导,且有:
\[
\frac{dy}{dx}=\frac{dy}{du}\cdot \frac{du}{dx}
\]
\end{theorem}

复合求导法则是接下去三种求导法则(反函数求导、隐函数求导、参数式函数求导)的基础。

%============================================================
\subsection{拓展讨论:反函数的求导}

\begin{theorem}[反函数求导定理]
若函数$y=f\left( x \right) $在某区间内单调可导,且$y'\ne 0$,则它的反函数$x=f^{-1}\left( y \right) $在对应区间内也可导,且有:
\[
f'\left( x \right) =\frac{1}{\left( f^{-1} \right) '\left( y \right)} \quad \text{或写成} \quad \frac{dy}{dx}=\frac{1}{\frac{dx}{dy}}
\]
\end{theorem}

\begin{proof}
\[
f'\left( x \right) =\underset{\Delta x\rightarrow 0}{\lim}\frac{\Delta y}{\Delta x}=\underset{\Delta x\rightarrow 0}{\lim}\frac{1}{\frac{\Delta x}{\Delta y}}=\frac{1}{\underset{\Delta y\rightarrow 0}{\lim}\frac{\Delta x}{\Delta y}}=\frac{1}{\left( f^{-1} \right) '\left( y \right)}
\]
\end{proof}

~

\begin{example}
设$y=\mathrm{arc}\sin x$,求$y'$。
\end{example}

解:
\[
\frac{dy}{dx}=\frac{1}{\frac{dx}{dy}}=\frac{1}{\left( \sin y \right) '}=\frac{1}{\cos y}=\frac{1}{\sqrt{1-\sin ^2y}}=\frac{1}{\sqrt{1-x^2}}
\]

%============================================================
\subsection{拓展讨论:隐函数的求导}

\begin{theorem}[一元隐函数求导定理]
若方程$F\left( x,y \right) =0$确定唯一的单值连续可导函数$y=f\left( x \right) $,则有:
\[
\frac{dy}{dx}=-\frac{F_x}{F_y}
\]
其中:
\begin{itemize}
    \item $F_x$:表示$F\left( x,y \right) =0$对$x$求导,$x$是自变量,$y$是常量;
    \item $F_y$:表示$F\left( x,y \right) =0$对$y$求导,$y$是自变量,$x$是常量。
\end{itemize}
\end{theorem}

\begin{tcolorbox}
该定理在多元函数微积分中证明,这里只需知道如何计算一元隐函数的导数。
\end{tcolorbox}

~

\begin{example}
假设方程$e^y=e-xy$确定了函数$y=f\left( x \right) $,求$y'$。
\end{example}

解:

令$F\left( x,y \right) :=e^y-e+xy=0$,得:
\begin{align*}
&\because \begin{cases}
	F_x=y\\
	F_y=e^y+x\\
\end{cases} \\
&\therefore y'=-\frac{F_x}{F_y}=-\frac{y}{e^y+x}
\end{align*}

%============================================================
\subsection{拓展讨论:参数式函数的求导}

\begin{theorem}[参数函数求导定理]
若参数方程
\begin{align*}
\begin{cases}
	x=x\left( t \right)\\
	y=y\left( t \right)\\
\end{cases}
\end{align*}
可导,$x=x\left( t \right) $存在可导的反函数,且$x'\left( t \right) \ne 0$,则由该参数方程确定的函数$y=f\left( x \right) $可导,且有:
\[
\frac{dy}{dx}=\frac{y'\left( t \right)}{x'\left( t \right)}
\]
\end{theorem}

~

\begin{example}
若有摆线
\begin{align*}
\begin{cases}
	x=t-\sin t\\
	y=1-\cos t\\
\end{cases}
\end{align*}
求$t=\pi /2$处的切线方程。
\end{example}

解:

切线方程$y=f'\left( x_0 \right) x+\left[ f\left( x_0 \right) -f'\left( x_0 \right) x_0 \right] $,可得:
\begin{align*}
&\left. \frac{dy}{dx} \right|_{t=\pi /2}=\left. \frac{dy}{dt}/\frac{dx}{dt} \right|_{t=\pi /2}=\left. \frac{\sin t}{1-\cos t} \right|_{t=\pi /2}=1 \\
&y=1\cdot x+\left[ \left( 1-\cos \frac{\pi}{2} \right) -1\cdot \left( \frac{\pi}{2}-1 \right) \right] =x+2-\frac{\pi}{2}
\end{align*}

%============================================================
\subsection{拓展讨论:高阶导数}

\begin{definition}[二阶导数]
设函数$f\left( x \right) $在点$x_0$的某邻域$N\left( x_0 \right) $内可导,若极限
\[
\underset{\Delta x\rightarrow 0}{\lim}\frac{f'\left( x_0+\Delta x \right) -f'\left( x_0 \right)}{\Delta x}
\]
存在,则称该极限值为{\bf $f\left( x \right) $在点$x_0$处的二阶导数},记作:
\[
f''\left( x_0 \right) \quad \text{或} \quad \left. y'' \right|_{x=x_0} \quad \text{或} \quad \left. \frac{d^2y}{dx^2} \right|_{x=x_0}
\]
\end{definition}

同样,我们可以定义三阶导数、四阶导数、直至$n$阶导数。

%============================================================
\subsection{习题}

\begin{example}[综合运用6,难度:$\star $]
已知函数$f\left( x \right) $满足$f\left( x \right) =f'\left( \frac{\pi}{4} \right) \sin x-\cos x$,求$f\left( x \right) $在$x=\frac{\pi}{4}$处的导数。
\end{example}

解:

对函数求导:
\begin{align*}
&f'\left( x \right) =f'\left( \frac{\pi}{4} \right) \cos x+\sin x \\
&f'\left( \frac{\pi}{4} \right) =f'\left( \frac{\pi}{4} \right) \cdot \frac{\sqrt{2}}{2}+\frac{\sqrt{2}}{2}
\end{align*}

\begin{tcolorbox}
本题没有难度。
\end{tcolorbox}




