\section{导数在研究函数中的应用}

本节要点:
\begin{itemize}
    \item 掌握函数单调性的判断方法;
    \item 掌握函数极值的求解方法。
\end{itemize}

%============================================================
\subsection{函数的单调性}

\begin{theorem}
若$f\left( x \right) \in D\left( a,b \right) $,则:
\begin{itemize}
    \item $f'\left( x \right) \geqslant 0\left( \leqslant 0 \right) \Leftrightarrow f\left( x \right) $在$\left( a,b \right) $上单调增(减);
    \item $f'\left( x \right) >0\left( <0 \right) \Leftrightarrow f\left( x \right) $在$\left( a,b \right) $上严格单调增(减)。
\end{itemize}
\end{theorem}

\begin{corollary}
若$f\left( x \right) \in C\left[ x_0,x \right] \cap D\left( x_0,x \right) $,且$f\left( x_0 \right) =0$,则:
\begin{itemize}
    \item 当$x>x_0$时$f'\left( x \right) >0$,则$f\left( x \right) >0$;
    \item 当$x>x_0$时$f'\left( x \right) <0$,则$f\left( x \right) <0$。
\end{itemize}
\end{corollary}

推论要表达的是当我们知道函数的某个起点和趋势,我们就可以判断函数的在该点后的值域。

%============================================================
\subsection{函数的极值与最大(小)值}

\begin{definition}[极值和极值点]
设函数$f\left( x \right) $在区间$D$上有定义,若$x_0\in D$存在邻域$N\left( x_0,\delta \right) \subseteq D$,使得$\forall x\in N$有:
\[
f\left( x \right) <f\left( x_0 \right) \quad \text{或} \quad f\left( x \right) >f\left( x_0 \right)
\]
则称$f\left( x_0 \right) $为$f\left( x \right) $在邻域$N$上的一个{\bf 极大值}(或{\bf 极小值},统称为{\bf 极值}),而点$x_0$称为$f\left( x \right) $在邻域$N$上的{\bf 极大值点}(或{\bf 极小值点},统称为{\bf 极值点})。如果$f\left( x \right) $在区间$D$上可导,则极值点$x_0$处必有$f'\left( x_0 \right) =0$,但$f'\left( x_0 \right) =0$并不代表$x_0$为极值点,我们称$f'\left( x_0 \right) =0$时的$x_0$为$f\left( x_0 \right) =0$的{\bf 驻点}。如果$f\left( x_0 \right) =0$在$x_0$处不可导,则$x_0$也有可能是极值点。
\end{definition}

\begin{theorem} \label{th_5_3_1}
设$f\left( x \right) =0$在点$x_0$处连续,在$N\left( \hat{x}_0,\delta \right) $内可导,则:
\begin{itemize}
    \item 若$\forall x\in \left( x_0-\delta ,x_0 \right) $有$f'\left( x \right) <0$,$\forall x\in \left( x_0,x_0+\delta \right) $有$f'\left( x \right) >0$,则$f\left( x_0 \right) $为极小值;
    \item 若$\forall x\in \left( x_0-\delta ,x_0 \right) $有$f'\left( x \right) >0$,$\forall x\in \left( x_0,x_0+\delta \right) $有$f'\left( x \right) <0$,则$f\left( x_0 \right) $为极大值;
    \item 若$\forall x\in N\left( \hat{x}_0,\delta \right) $有$f'\left( x \right) $恒为正或恒为负,$f\left( x_0 \right) $为非极值。
\end{itemize}
\end{theorem}

\begin{theorem} \label{th_5_3_2}
设$f\left( x \right) =0$在点$x_0$处存在二阶导数,且$f'\left( x \right) =0$,则:
\begin{itemize}
    \item 若$f''\left( x \right) >0$,则$f\left( x_0 \right) $为极小值;
    \item 若$f''\left( x \right) <0$,则$f\left( x_0 \right) $为极大值;
    \item 若$f''\left( x \right) =0$,不能判定。
\end{itemize}
\end{theorem}

\begin{theorem} \label{th_5_3_3}
设$f\left( x \right) =0$在点$x_0$处存在二阶及以上导数,且
\begin{align*}
&f'\left( x_0 \right) =f''\left( x_0 \right) =\cdots =f^{\left( n-1 \right)}\left( x_0 \right) =0 \\
&f^{\left( n \right)}\left( x_0 \right) \ne 0
\end{align*}
则:
\begin{itemize}
    \item $n$为偶数时,若$f^{\left( n \right)}\left( x_0 \right) >0$则$f\left( x_0 \right) $为极小值,若$f^{\left( n \right)}\left( x_0 \right) <0$则$f\left( x_0 \right) $为极大值;
    \item $n$为奇数时,$f\left( x_0 \right) $非极值。
\end{itemize}
\end{theorem}

上述3个定理是对微分中值定理中的费马定理的推广,费马定理给出的是极值点的必要条件,这里给出了极值的充要条件。但必须注意,这些定理的前提是可导,不可导点也是潜在的极值点,需要判断。

第1个定理可以从几何上理解,通过描述一阶导数的走势判断极值。第2个定理通过二阶导数定量化地描述了函数走势。第3个定理是第2个定理的推广。

%============================================================
\subsection{拓展讨论:目标函数优化}

极值是一个小范围内的最值,如果将考察区域扩大,则对极值的考察就扩展为对最值的考察。最值点将会出现在极值点、不可导点和边界点上。工程上有很多类似考察最值的问题,数学上归结为{\bf 求解目标函数的最值问题},有时又称为{\bf 目标函数的最优化}。

若$f\left( x \right) $在$\left[ a,b \right] $内连续,$\left( a,b \right) $内存在有限个不可导点,则首先根据连续函数的性质,必然存在最值,且两个端点$a,b$、驻点和不可导点$x_i$都是最值可能的点,将这些点的函数值计算就可以判断最值,或者根据一阶导数判断最值。

~

\begin{example}
若$f\left( x \right) =\sqrt[3]{\left( x^2-2x \right) ^2}$,求$\left[ 0,3 \right] $上的最值。
\end{example}

解:

易得$f\left( x \right) $连续,考察一阶导数:
\[
f'\left( x \right) =\frac{4}{3}\frac{x-1}{\sqrt[3]{x^2-2x}}
\]
有$x=1$为驻点,$x=2$为不可导点,于是端点、驻点、不可导点集合:
\[
\left\{ 0,3,1,2 \right\}
\]
分别求解:
\[
f\left( 0 \right) =0 \quad f\left( 3 \right) =\sqrt[3]{9} \quad f\left( 1 \right) =1 \quad f\left( 2 \right) =0
\]
于是可得最值:
\[
f_{\min}=0 \quad f_{\max}=\sqrt[3]{9}
\]

~

\begin{example}
若要做一个容积为$V_0$的圆柱形储罐,怎样设计用料最省。
\end{example}

解:

圆柱形储罐容积$V=\pi r^2h$,用料为表面积:
\[
S\left( r \right) =2\pi r^2+2\pi rh=2\pi r^2+2\pi r\frac{V}{\pi r^2}=2\pi r^2+2\frac{V_0}{r},    r\in \left( 0,+\infty \right)
\]
即求$S\left( r \right) $的最值,考察一阶导数:
\[
S'\left( r \right) =4\pi r-2V_0\frac{1}{r^2}=\frac{4\pi r^3-2V_0}{r^2}
\]
在$\left( 0,+\infty \right) $上可导且只有一个驻点
\begin{align*}
&\because \frac{4\pi r^3-2V_0}{r^2}=0 \\
&\therefore r_0=\sqrt[3]{\frac{V_0}{2\pi}}
\end{align*}
考察二阶导数$S''\left( r \right) =\frac{4\pi r^3+4V_0}{r^3}>0$,所以该驻点为极小值点。由于$S\left( r \right) $在$\left( 0,+\infty \right) $上只有一个极小值点,且无不可导点,无端点,所以$r_0$为$S\left( r \right) $的最小值点,此时:
\[
h_0=\frac{V_0}{\pi {r_0}^2}=2\sqrt[3]{\frac{V_0}{2\pi}}=2r_0
\]
即储罐高和底面直径相等时,用料最少。

~

\begin{example}
假设一个物理试验中,一共进行了$n$次测量,得到$x_1,x_2,\cdots ,x_n$,若用$\bar{x}$表示测量结果,问$\bar{x}$为多少使得测量的总平方误差
\[
TSE\left( \bar{x} \right) =\sum_{i=1}^n{\left( \bar{x}-x_i \right) ^2}
\]
最小。
\end{example}

解:

首先易得$TSE\left( x \right) $在$\left( -\infty ,+\infty \right) $连续,考察一阶和二阶导数:
\begin{align*}
&TSE'\left( x \right) =2\sum_{i=1}^n{\left( x-x_i \right)}=2nx-2\sum_{i=1}^n{x_i} \\
&TSE''\left( x \right) =2n>0
\end{align*}
可得,当
\[
\bar{x}=\frac{1}{n}\sum_{i=1}^n{x_i}
\]
时,$TSE\left( x \right) $有极小值,且是最小值。

这里可以看出,当取算术平均值时,总平方误差最小。所以用算术平均值代替测量值,在总平方误差的角度是最可靠的。

%============================================================
\subsection{拓展讨论:凸函数和拐点}

\begin{definition}[凸函数]
设$f\left( x \right) \in C\left[ a,b \right] $,若$\forall x_1,x_2\in \left( a,b \right) ,x_1\ne x_2$和$\forall t_1,t_2>0,t_1+t_2=1$:
\begin{itemize}
    \item 若$f\left( t_1x_1+t_2x_2 \right) <t_1f\left( x_1 \right) +t_2f\left( x_2 \right) $,则称$f\left( x \right) $在$\left( a,b \right) $上{\bf 下凸};
    \item 若$f\left( t_1x_1+t_2x_2 \right) >t_1f\left( x_1 \right) +t_2f\left( x_2 \right) $,则称$f\left( x \right) $在$\left( a,b \right) $上{\bf 上凸};
\end{itemize}
通常我们将下凸函数称为{\bf 凸函数}。
\end{definition}

\begin{theorem}
若$f\left( x \right) $在$\left[ a,b \right] $内连续,$\left( a,b \right) $内二阶可导且恒有$f''\left( x \right) >0$(或$f''\left( x \right) <0$),则$f\left( x \right) $在$\left( a,b \right) $内下凸(或上凸)。
\end{theorem}

几何上,凸函数表示$\left[ a,b \right] $上任取两点作连线,$f\left( x \right) $都在连线之下。注意这里取点的任意性。

\begin{definition}[拐点]
设$f\left( x \right) $在$N\left( x_0 \right) $内连续,若在$x_0$的左右两侧凸性相反,则称$x_0$为$f\left( x \right) $的{\bf 拐点}。
\end{definition}

\begin{theorem}
若$f\left( x \right) $在$\left( a,b \right) $内二阶可导,$x_0\in \left( a,b \right) $:
\begin{itemize}
    \item 若$x_0$是$f\left( x \right) $的一个拐点,则有$f''\left( x_0 \right) =0$;
    \item 若$f''\left( x_0 \right) =0$且$f''\left( {x_0}^+ \right) \cdot f''\left( {x_0}^- \right) <0$(即$x_0$两侧$f''\left( x \right) $异号),则$x_0$是$f\left( x \right) $的一个拐点。
\end{itemize}
\end{theorem}

%============================================================
\subsection{拓展讨论:琴生(Jensen)不等式}

\begin{tcolorbox}
反过来我们用凸函数的定义可以获得一个比较重要的不等式。
\end{tcolorbox}

\begin{definition}[琴生(Jensen)不等式]
若$f\left( x \right) $在$\left( a,b \right) $内下凸,则对于
\begin{align*}
&\forall x_1,x_2,\cdots ,x_n\in \left( a,b \right) \\
&\forall t_1,t_2,\cdots ,t_n>0 \\
&\sum_{i=1}^n{t_i}=1
\end{align*}
必有:
\[
f\left( \sum_{i=1}^n{t_ix_i} \right) <\sum_{i=1}^n{\left[ t_if\left( x_i \right) \right]}
\]
其中$x_1,x_2,\cdots ,x_n$不全相等。
特别的,当$t_1=t_2=\cdots =t_n=1/n$时,有:
\[
f\left( \frac{x_1+x_2+\cdots +x_n}{n} \right) <\frac{f\left( x_1 \right) +f\left( x_2 \right) +\cdots +f\left( x_n \right)}{n}
\]
\end{definition}

%============================================================
\subsection{拓展讨论:渐近线}

\begin{definition}[函数的渐近线]
对于曲线$f\left( x \right) $,若存在直线$y=kx+b$使得:
\[
\underset{x\rightarrow +\infty}{\lim}\left[ f\left( x \right) -y\left( x \right) \right] =0 \quad \text{或} \quad \underset{x\rightarrow -\infty}{\lim}\left[ f\left( x \right) -y\left( x \right) \right] =0
\]
则称$y=kx+b$为{\bf $f\left( x \right) $的渐近线},特别的当$k=0$时,称$y=b$为{\bf $f\left( x \right) $的水平渐近线}。若曲线$f\left( x \right) $有:
\[
\underset{x\rightarrow {x_0}^-}{\lim}f\left( x \right) =\infty  \quad \text{或} \quad \underset{x\rightarrow {x_0}^+}{\lim}f\left( x \right) =\infty
\]
则称$x=x_0$为{\bf $f\left( x \right) $的垂直渐近线}。
\end{definition}

渐近线判断方法:
\begin{enumerate}
    \item 考察$\left( -\infty ,+\infty \right) $上未定义点$x_0$处的极限$\underset{x\rightarrow x_0}{\lim}f\left( x \right) $,若为$\infty $,说明有垂直渐近线$x=x_0$。
    \item 计算$k_1=\underset{x\rightarrow +\infty}{\lim}\frac{f\left( x \right)}{x}$、$k_2=\underset{x\rightarrow -\infty}{\lim}\frac{f\left( x \right)}{x}$,和$b_1=\underset{x\rightarrow +\infty}{\lim}\left[ f\left( x \right) -kx \right] $、$b_2=\underset{x\rightarrow -\infty}{\lim}\left[ f\left( x \right) -kx \right] $,获得斜渐近线或水平渐近线。
\end{enumerate}

~

\begin{example}
判断逻辑斯蒂(Logistic)函数$f\left( x \right) =\frac{c}{1+be^{-ax}}$的渐近线。
\end{example}

解:

$f\left( x \right) $在$\left( -\infty ,+\infty \right) $内均有定义,所以没有垂直渐近线,考察斜渐近线和水平渐近线:
\begin{align*}
&k_{1,2}=\underset{x\rightarrow \pm \infty}{\lim}\frac{f\left( x \right)}{x}=\underset{x\rightarrow \pm \infty}{\lim}\frac{c}{\left( 1+be^{-ax} \right) x}=0 \\
&b_1=\underset{x\rightarrow +\infty}{\lim}\left[ f\left( x \right) -kx \right] =\underset{x\rightarrow +\infty}{\lim}\frac{c}{1+be^{-ax}}=c \\
&b_2=\underset{x\rightarrow -\infty}{\lim}\left[ f\left( x \right) -kx \right] =\underset{x\rightarrow -\infty}{\lim}\frac{c}{1+be^{-ax}}=0
\end{align*}
可见$f\left( x \right) =\frac{c}{1+be^{-ax}}$有两条渐近线:
\begin{align*}
&y=c \\
&y=0
\end{align*}

~

\begin{example}
考察$f\left( x \right) =\frac{\left( x-3 \right) ^2}{4\left( x-1 \right)}$的渐近线。
\end{example}

解:

显然,$x=1$处未定义,考察:
\[
\underset{x\rightarrow 1}{\lim}\frac{\left( x-3 \right) ^2}{4\left( x-1 \right)}=\infty
\]
可得$x=1$为$f\left( x \right) $的一条垂直渐近线。再考察斜渐近线和水平渐近线:
\begin{align*}
&k_{1,2}=\underset{x\rightarrow \pm \infty}{\lim}\frac{f\left( x \right)}{x}=\underset{x\rightarrow \pm \infty}{\lim}\frac{\left( x-3 \right) ^2}{4\left( x-1 \right) x}=\frac{1}{4} \\
&b_{1,2}=\underset{x\rightarrow \pm \infty}{\lim}\left[ f\left( x \right) -kx \right] =\underset{x\rightarrow \pm \infty}{\lim}\left[ \frac{\left( x-3 \right) ^2}{4\left( x-1 \right)}-\frac{x}{4} \right] =-\frac{5}{4}
\end{align*}
可得 只有一条斜渐近线:
\[
y=\frac{1}{4}x-\frac{5}{4}
\]




