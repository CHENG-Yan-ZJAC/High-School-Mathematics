\section{条件概率与全概率公式}

本节要点:
\begin{itemize}
    \item 掌握并理解条件概率的概念;
    \item 掌握并理解全概率公式;
    \item 深刻理解贝叶斯公式的哲学意义。
\end{itemize}

%============================================================
\subsection{条件概率}

\begin{definition}[条件概率]
在事件$A$发生的情况下,事件$B$发生的概率称为{\bf 条件概率},记作$P\left( B \middle| A \right) $,当$P\left( A \right) >0$时有:
\[
P\left( B \middle| A \right) := \frac{P\left( AB \right)}{P\left( A \right)}
\]
\end{definition}

\begin{tcolorbox}
$P\left( B \middle| A \right) $和$P\left( B \right) $的大小关系是不定的。
也就是说,事件$A$的发生并不一定有利于$B$,也可能没有影响。
切不能想当然地认为$P\left( B \middle| A \right) >P\left( B \right) $!
\end{tcolorbox}

\begin{theorem}[乘法公式]
若$P\left( A \right) >0$,则
\[
P\left( AB \right) =P\left( A \right) \left( B \middle| A \right)
\]
\end{theorem}

\begin{corollary}[乘法公式]
若$P\left( A_1A_2\cdots A_n \right) >0,n\geqslant 2$,则
\[
P\left( A_1A_2\cdots A_n \right) =P\left( A_1 \right) P\left( A_2 \middle| A_1 \right) P\left( A_3 \middle| A_1A_2 \right) \cdots P\left( A_n \middle| A_1A_2\cdots A_{n-1} \right)
\]
\end{corollary}

%============================================================
\subsection{全概率公式}

\begin{definition}[完备事件组]
若事件组$A_1,A_2,\cdots ,A_i,\cdots $两两互斥,且$A_1+A_2+\cdots +A_i+\cdots =\varOmega $,则$A_1,A_2,\cdots ,A_i,\cdots $构成样本空间$\varOmega $的一个{\bf 完备事件组},简称{\bf 完备组}。
\end{definition}

\begin{tcolorbox}
虽然教材上没提,但是我们还是需要理解完备组的概念,运用全概率的前提就是找到一个完备组。
\end{tcolorbox}

\begin{definition}[全概率公式]
若事件组$A_1,A_2,\cdots ,A_i,\cdots $构成样本空间$\varOmega $的一个完备组,则对于任何事件$B$有:
\[
P\left( B \right) =\sum_{i=1}^{\infty}{P\left( A_iB \right)}=\sum_{i=1}^{\infty}{\left[ P\left( A_i \right) \cdot P\left( B \middle| A_i \right) \right]}
\]
\end{definition}

完备组“瓜分”了整个样本空间,是事件对立关系的扩充,其意义在于将一个复杂的大问题分解为若干个简单的小问题。

全概率可以认为是条件概率的延申,如果事件组满足完备条件,则事件$B$的发生总是伴随某个组员的发生,或者说$B$是通过某个路径发生的。当$P\left( B \right) $的计算较为复杂时,我们可以构造一个合适的完备组,简化$P\left( B \right) $的计算。其次,该公式从数学上还可以理解为$P\left( B \right) $是$P\left( B \middle| A_i \right) $的加权和,权重为$P\left( A_i \right) $。最后还需注意,$P\left( B \right) \leqslant 1$,但$\sum_{i=1}^{\infty}{P\left( B \middle| A_i \right)}$几乎可能大于1。

\begin{tcolorbox}
全概率公式$P\left( B \right) =\sum_{i=1}^n{\left[ P\left( A_i \right) \cdot P\left( B \middle| A_i \right) \right]} $中,$P\left( B \middle| A_i \right) $可以认为是事件$B$在基$A_i$下的坐标分量,该分量越大,表示越可能发生。
\end{tcolorbox}

\begin{definition}[贝叶斯公式]
若事件$A_1,A_2,\cdots ,A_i,\cdots $构成样本空间$\varOmega $的一个完备事件组,且$P\left( A_i \right) >0$,若$B$为一随机事件,且$P\left( B \right) >0$,则:
\[
P\left( A_i \middle| B \right) =\frac{P\left( A_i \right) P\left( B \middle| A_i \right)}{\sum_{i=1}^{\infty}{\left[ P\left( A_i \right) \cdot P\left( B \middle| A_i \right) \right]}}=\frac{P\left( A_i \right) P\left( B \middle| A_i \right)}{P\left( B \right)}
\]
又称{\bf 逆概率公式}。通常,我们将$P\left( A_i \right) $称为{\bf 先验概率},将条件概率$P\left( A_i \middle| B \right) $称为{\bf 后验概率}。也可以结合向量内积理解。
\end{definition}

\begin{proof}
由条件概率的概念可得:
\[
P\left( A_i \middle| B \right) =\frac{P\left( A_iB \right)}{P\left( B \right)}
\]
结合全概率公式,且$P\left( A_iB \right) =P\left( BA_i \right) =P\left( A_i \right) P\left( B \middle| A_i \right) $。
\end{proof}

$P\left( A_i \right) $称为先验概率的原因是$A_i$作为一种假定而不依赖于$B$。$B$称为{\bf 证据},一种出现的客观事物。在证据出现前,$A_i$有自己的概率。当证据发现时,$A_i$的概率变成了$P\left( A_i \middle| B \right) $,所以我们称为后验概率。“先”和“后”指的是证据出现前和出现后。其中$P\left( B \middle| A_i \right) $描述了$A_i$和$B$的关系,机器学习中称为{\bf 似然(likehood)}。

全概率公式体现了由因求果,而贝叶斯公式反映了执果寻因的一种辩证思想,我们知道完备组员的概率$P\left( A_i \right) $和$B$在各个完备组员的条件概率$P\left( B \middle| A_i \right) $,就可以计算$B$大概会发生在哪个组员。很多情况下,我们知道各个原因发生的可能性,但是一旦结果出现,这些原因的可能性又会出现变化,贝叶斯公式揭示了两者之间的关系。

%============================================================
\subsection{习题}

\begin{example}[复习巩固3,难度:$\star $]
甲、乙两人向同一个目标各射击1次,已知甲命中目标的概率为0.6,乙命中目标的概率为0.5。已知目标至少被命中1次,求甲命中目标的概率。
\end{example}

解:

设甲命中目标为$A$,乙命中目标为$B$,目标至少被命中1次为$C$,由于甲乙相互独立,有:
\[
P\left( C \right) =P\left( A \right) +P\left( \bar{A}B \right) =P\left( A \right) +P\left( \bar{A} \right) \cdot P\left( B \right)
\]
甲命中目标的概率可表示为条件概率:
\begin{align*}
P\left( A \middle| C \right) &=\frac{P\left( AC \right)}{P\left( C \right)}=\frac{P\left( A \right) \cdot P\left( \bar{B} \right) +P\left( A \right) \cdot P\left( B \right)}{P\left( A \right) +P\left( \bar{A} \right) \cdot P\left( B \right)} \\
&=\frac{0.6\cdot 0.5+0.6\cdot 0.5}{0.6+0.4\cdot 0.5}=0.75
\end{align*}

拓展讨论:

$C$的概率可以有如下三种算法:
\begin{itemize}
    \item 甲命中乙不命中+甲不命中乙命中+甲乙都命中;
    \item 甲命中+甲不命中乙命中;
    \item 乙命中+乙不命中甲命中。
\end{itemize}
以上三种算法是三种完备组,易得这三种方法的结果是一样的。
\begin{align*}
P\left( C \right) &=P\left( A \right) +P\left( \bar{A}B \right) =P\left( A \right) +P\left( \bar{A} \right) \cdot P\left( B \right) \\
&=P\left( A \right) \cdot \left[ P\left( B \right) +P\left( \bar{B} \right) \right] +P\left( \bar{A} \right) \cdot P\left( B \right) \\
P\left( C \right) &=P\left( B \right) +P\left( \bar{B}A \right) =P\left( B \right) +P\left( \bar{B} \right) \cdot P\left( A \right) \\
&=P\left( B \right) \left[ P\left( A \right) +P\left( \bar{A} \right) \right] +P\left( \bar{B} \right) \cdot P\left( A \right) \\
P\left( C \right) &=P\left( A\bar{B} \right) +P\left( \bar{A}B \right) +P\left( AB \right) \\
&=P\left( A \right) \cdot P\left( \bar{B} \right) +P\left( \bar{A} \right) \cdot P\left( B \right) +P\left( A \right) \cdot P\left( B \right)
\end{align*}

\begin{tcolorbox}
本题关键在于理解$P\left( C \right) ,P\left( AC \right) $。
\end{tcolorbox}

~

\begin{example}[复习巩固4,难度:$\star $]
甲乙两个箱子中各装有10个球,其中甲箱中有5个红球、5个白球,乙箱中有8个红球、2个白球。掷一枚质地均匀的骰子,如果点数为1或2,从甲箱子中随机摸出1个球;如果点数为3,4,5,6,从乙箱子中随机摸出1个球。
求摸到红球的概率。
\end{example}

解:

设骰子点数为1或2为$A$,从甲箱子中摸出红球为$B$,从乙箱子中摸出红球为$C$,则:
\begin{align*}
P&=P\left( A \right) \cdot P\left( \bar{B} \right) +P\left( A \right) \cdot P\left( C \right) \\
&=\frac{2}{6}\cdot \frac{5}{10}+\frac{4}{6}\cdot \frac{8}{10}=\frac{7}{10}
\end{align*}

拓展讨论:

红球要么从甲箱要么从乙箱中获得,构成一个完备组,而骰子的出现可以认为是给甲箱和乙箱加了一个加权系数,而且加权系数和为1。

\begin{tcolorbox}
本题运用全概率公式即可。
\end{tcolorbox}

~

\begin{example}[复习巩固5,难度:$\star $]
在A,B,C三个地区爆发了流感,这三个地区分别有6\%,5\%,4\%的人患了流感。假设这三个地区的人口数的比为5:7:8,现从这三个地区中任意选取一个人。
\begin{enumerate}
    \item 求这个人患流感的概率;
    \item 如果此人患流感,求此人选自A地区的概率。
\end{enumerate}
\end{example}

解:

显然地区A、B、C构成完备组。

(1)
\[
P\left( A \right) =\frac{5}{5+7+8}\cdot 6\%+\frac{7}{5+7+8}\cdot 5\%+\frac{8}{5+7+8}\cdot 4\%=4.85\%
\]

(2)
\[
P=\frac{\frac{5}{5+7+8}\cdot 6\%}{P\left( A \right)}=\frac{0.015}{0.0485}=0.309278
\]

\begin{tcolorbox}
本题考查条件概率,不难。
\end{tcolorbox}

~
\begin{example}[综合运用7,难度:$\star $]
一批产品共有100件,其中5件为不合格品。收货方从中不放回地随机抽取产品进行检验,并按以下规则判断是否接受这批产品:如果抽检的第1件产品不合格,则拒绝整批产品;如果抽检的第1件产品合格,则再抽1件,如果抽检的第2件产品合格,则接受整批产品,否则拒绝整批产品。求这批产品被拒绝的概率。
\end{example}

解:

易得,这批产品要么整批接收,要么整批拒绝,显然整批接收的条件为两次抽检均合格,比较好计算,有:
\[
1-P\left( A \right) =1-\frac{95}{100}\cdot \frac{94}{99}=0.0979798
\]

\begin{tcolorbox}
本题用补集清晰简单。
\end{tcolorbox}

%============================================================
\subsection{拓展习题}

\begin{example}
设10件产品中存在2件次品,现不放回地抽取2次,每次1件,求第2件是次品的概率。
\end{example}

解:

显然第2件是否为次品和第1件为否是次品有关,是典型的条件概率。更进一步,第1件产品的合格和次品两个事件可以构成完备组,所以可以用全概率公式求解。

设第1次抽取为次品为事件$A$,于是$A,\bar{A}$可视为完备组,且易得$P\left( A \right) =\frac{2}{10},P\left( \bar{A} \right) =\frac{8}{10}$。再设第2次抽到次品为事件$B$,于是$P\left( B \right) $可以用这个完备组分解:
\[
P\left( B \right) =P\left( A \right) P\left( B \middle| A \right) +P\left( \bar{A} \right) P\left( B \middle| \bar{A} \right)
\]
其中:
\begin{itemize}
    \item $P\left( B \middle| A \right) $:表示第1次抽到次品时第2次也抽到次品的概率,易得$\frac{1}{9}$;
    \item $P\left( B \middle| \bar{A} \right) $:表示第1次抽到合格时第2次抽到次品的概率,易得$\frac{2}{9}$。
\end{itemize}
最终得到:
\[
P\left( B \right) =\frac{1}{5}\cdot \frac{1}{9}+\frac{4}{5}\cdot \frac{2}{9}=\frac{1}{5}
\]

解二,不用全概率的概念,直接使用古典概型的方法,抽取2件排序,共$P_{10}^{2}$种,第2件是次品有两种互不相同可能:
\begin{itemize}
    \item 第1次抽到次品时第2次也抽到次品,$C_{2}^{1}\cdot C_{1}^{1}$种;
    \item 第1次抽到合格时第2次抽到次品,$C_{8}^{1}\cdot C_{2}^{1}$种。
\end{itemize}
于是:
\[
P=\frac{C_{2}^{1}\cdot C_{1}^{1}+C_{8}^{1}\cdot C_{2}^{1}}{P_{10}^{2}}=\frac{18}{90}=\frac{1}{5}
\]

~

\begin{example}
假设血清甲蛋白法测定肝癌阳性准确率为95\%、阴性准确率为90\%,若已知社会人群的患癌率为0.04\%,若某人检测阳性,分析其实际患癌的概率。
\end{example}

解:

设事件$A$表示被检者实际患癌,事件$B$ 表示用该测定法判定阳性,由题干可知:
\begin{align*}
&P\left( A \right) =0.0004 \\
&P\left( B \middle| A \right) =0.95 \\
&P\left( \bar{B} \middle| \bar{A} \right) =0.90
\end{align*}
由于$A,\bar{A}$构成完备组,则有:
\begin{align*}
P\left( A \middle| B \right) &=\frac{P\left( A \right) P\left( B \middle| A \right)}{P\left( A \right) P\left( B \middle| A \right) +P\left( \bar{A} \right) P\left( B \middle| \bar{A} \right)} \\
&=\frac{0.0004\cdot 0.95}{0.0004\cdot 0.95+0.9996\cdot \left( 1-0.9 \right)} =0.00378712
\end{align*}

一方面可见即便测定阳性,实际患癌也不到1\%,但比一般人群已经高了$0.3787/0.04=9.4675$,将近10倍。另一方面,通过贝叶斯公式我们了解到,血清甲蛋白测定法不是判断患癌的概率,而是判断比正常人患癌概率高了多少。

\begin{tcolorbox}
该例子很具典型性。首先0.04\%的患癌概率是作为一种先天的经验存在。测试结果则是一种证据。证据的出现可以让我们对这个先天经验进行修正。由于这个证据本身带有概率,而非100\%,所以它实际上是提高了患癌概率。
\end{tcolorbox}

~

\begin{example}
某水文站测量水流量,令$B$为测量不准的事件,并总结了几个测量不准的原因$A_i$,原因的发生概率$P\left( A_i \right) $,以及某个原因发生时测量不准的概率$P\left( B \middle| A_i \right) $,如下表,分析哪几个原因主要导致了测量不准。
\begin{table}[h]
    \centering
    \begin{tabular}{cccc}
        \toprule
        测不准的原因$A_i$ & $P\left( A_i \right) $ & $P\left( B \middle| A_i \right) $ & $P\left( A_i \middle| B \right) $\\
        \midrule
        流速计故障 & 0.01 & 0.90 & 0.05\\
        绕道设备误差 & 0.05 & 0.80 & 0.24\\
        水中有杂质 & 0.04 & 0.10 & 0.02\\
        气象条件不佳 & 0.10 & 0.30 & 0.18\\
        人员配合问题 & 0.10 & 0.80 & 0.47\\
        其他 & 0.70 & 0.01 & 0.04\\
        \bottomrule
    \end{tabular}
\end{table}
\end{example}

解:

“那几个原因主要导致了测量不准”也即测量不准的情况下,各个原因的发生概率$P\left( A_i \middle| B \right) $,可以用贝叶斯公式解决。
\[
P\left( A_i \middle| B \right) =\frac{P\left( A_i \right) P\left( B \middle| A_i \right)}{P\left( B \right)}=\frac{P\left( A_i \right) P\left( B \middle| A_i \right)}{\sum_{i=1}^{\infty}{P\left( A_i \right) P\left( B \middle| A_i \right)}}
\]
计算结果如上表最右列。可见,单看每个已知的明确的原因,它们的发生概率都不大,但是一旦问题出现,最有可能是人员配合导致。所以若要提高测量准确度,首先得从提高人员配合度着手。

~

\begin{example}
甲箱中有2个白球4个黑球,乙箱中有6个白球2个黑球,现从这两箱中各任取一球,再从所取出的两球中任取一球,求最终该球是白球的概率。
\end{example}

解:

令从甲箱中取得白球为事件$A$,从乙箱中取得白球为事件$B$,易得甲乙两个箱子中取球为独立事件,于是:
\begin{align*}
&P\left( A \right) =\frac{1}{3},P\left( B \right) =\frac{3}{4} \\
&P\left( AB \right) =\frac{1}{4},P\left( A\bar{B} \right) =\frac{1}{12},P\left( \bar{A}B \right) =\frac{1}{2},P\left( \bar{A}\bar{B} \right) =\frac{1}{6}
\end{align*}
且$AB,\bar{A}B,A\bar{B},\bar{A}\bar{B}$构成完备组,于是可以用全概率公式得到:
\begin{align*}
P\left( C \right) =&P\left( AB \right) P\left( C \middle| AB \right) +P\left( A\bar{B} \right) P\left( C \middle| A\bar{B} \right) + \\
&P\left( \bar{A}B \right) P\left( C \middle| \bar{A}B \right) +P\left( \bar{A}\bar{B} \right) P\left( C \middle| \bar{A}\bar{B} \right)
\end{align*}
而易得:
\begin{align*}
&P\left( C \middle| AB \right) =1 \\
&P\left( C \middle| A\bar{B} \right) =P\left( C \middle| \bar{A}B \right) =\frac{1}{2} \\
&P\left( C \middle| \bar{A}\bar{B} \right) =0
\end{align*}
于是:
\[
P\left( C \right) =\frac{1}{4}+\frac{1}{12}\cdot \frac{1}{2}+\frac{1}{2}\cdot \frac{1}{2}=\frac{13}{24}
\]




