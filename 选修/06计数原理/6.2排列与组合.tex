\section{排列与组合}

本节要点:
\begin{itemize}
    \item 掌握排列的概念及其计算公式;
    \item 掌握组合的概念及其计算公式。
\end{itemize}

~

\begin{definition}[排列和组合]
从$n$个元素中任取$m$个元素($m\leqslant n$)排成一有序列,称{\bf 从$n$个元素中取出$m$个元素的一个排列},所有排列的个数称作{\bf 排列数},记作$P_{n}^{m}$,有:
\[
P_{n}^{m}:=\frac{n!}{\left( n-m \right) !}=n\left( n-1 \right) \left( n-2 \right) \cdots \left[ n-\left( m-1 \right) \right]
\]
从$n$个元素中任取$m$个元素($m\leqslant n$)组成一无序组,称{\bf 从$n$个元素中取出$m$个元素的一个组合},所有组合的个数称作{\bf 组合数},记作$C_{n}^{m}$,有:
\[
C_{n}^{m}:=\frac{n!}{\left( n-m \right) !\cdot m!}=\frac{P_{n}^{m}}{m!}
\]
特别地,$P_{n}^{1}=C_{n}^{1}=n,P_{n}^{n}=n!,C_{n}^{n}=1$。
\end{definition}

排列组合均是对“从$n$个元素中抽取$m$个元素”的计算,排列突出有序,组合突出无序。排列组合的关系还可以这么理解,当$m$个元素排列完成后,若不考虑其顺序,就是组合,由于每个排列有$P_{m}^{m}$种不同的顺序,所以:
\[
P_{n}^{m}=C_{n}^{m}\cdot P_{m}^{m}=C_{n}^{m}\cdot m!
\]

\begin{theorem}
组合公式有以下性质:
\begin{align*}
&C_{n}^{m}=C_{n}^{n-m} \\
&C_{n+1}^{m}=C_{n}^{m}+C_{n}^{m-1}
\end{align*}
\end{theorem}




