\section{空间向量及其运算的坐标表示}

本节要点:
\begin{itemize}
    \item 掌握向量运算的坐标表示。
\end{itemize}

%============================================================
\subsection{空间直角坐标系}

和平面向量一样,当我们定义了空间直角坐标系后,可以得到空间向量的3个等价表示方法:
\begin{itemize}
    \item $\overrightarrow{AB}$:表示三维空间的有向线段;
    \item $x\boldsymbol{i}+y\boldsymbol{j}+z\boldsymbol{k}$:表示{\it xyz}坐标系中的有向线段;
    \item $\left( x,y,z \right) $:表示{\it xyz}坐标系中的点。
\end{itemize}

\begin{tcolorbox}
这里要注意一点,直角坐标系的右手性是人为规定的。
\end{tcolorbox}

%============================================================
\subsection{空间向量运算的坐标系表示}

加法、数乘、内积的坐标表示如下:
\begin{align*}
&\boldsymbol{a}+\boldsymbol{b}=\left( x_{\boldsymbol{a}}+x_{\boldsymbol{b}} \right) \boldsymbol{i}+\left( y_{\boldsymbol{a}}+y_{\boldsymbol{b}} \right) \boldsymbol{j}+\left( z_{\boldsymbol{a}}+z_{\boldsymbol{b}} \right) \boldsymbol{k} \\
&\lambda \boldsymbol{a}=\lambda x_{\boldsymbol{a}}\boldsymbol{i}+\lambda y_{\boldsymbol{a}}\boldsymbol{j}+\lambda z_{\boldsymbol{a}}\boldsymbol{k} \\
&\boldsymbol{a}\cdot \boldsymbol{b}=x_{\boldsymbol{a}}\cdot x_{\boldsymbol{b}}+y_{\boldsymbol{a}}\cdot y_{\boldsymbol{b}}+z_{\boldsymbol{a}}\cdot z_{\boldsymbol{b}}=\left| \boldsymbol{a} \right|\left| \boldsymbol{b} \right|\cos \alpha
\end{align*}

%============================================================
\subsection{习题}

\begin{example}[拓广探索9,难度:$\star \star $]
$\left\{ \boldsymbol{a},\boldsymbol{b},\boldsymbol{c} \right\} $是空间的一个单位正交基底,向量$\boldsymbol{p}=\boldsymbol{a}+2\boldsymbol{b}+3\boldsymbol{c}$,$\left\{ \boldsymbol{a}+\boldsymbol{b},\boldsymbol{a}-\boldsymbol{b},\boldsymbol{c} \right\} $是空间的另一个基底,用$\left\{ \boldsymbol{a}+\boldsymbol{b},\boldsymbol{a}-\boldsymbol{b},\boldsymbol{c} \right\} $表示向量$\boldsymbol{p}$。
\end{example}

解:

$\boldsymbol{p}$在新坐标系下可表示为:
\[
\boldsymbol{p}=x\left( \boldsymbol{a}+\boldsymbol{b} \right) +y\left( \boldsymbol{a}-\boldsymbol{b} \right) +z\boldsymbol{c}=\left( x+y \right) \boldsymbol{a}+\left( x-y \right) \boldsymbol{b}+z\boldsymbol{c}
\]
于是:
\begin{align*}
&\begin{cases}
	x+y=1\\
	x-y=2\\
	z=3\\
\end{cases}\Rightarrow \begin{cases}
	x=\frac{3}{2}\\
	y=-\frac{1}{2}\\
	z=3\\
\end{cases} \\
&\boldsymbol{p}=\frac{3}{2}\left( \boldsymbol{a}+\boldsymbol{b} \right) -\frac{1}{2}\left( \boldsymbol{a}-\boldsymbol{b} \right) +3\boldsymbol{c}
\end{align*}

\begin{tcolorbox}
本题其实是对空间向量基本定理的拓展,任意向量在不同基下有不同的表示,但向量还是那个向量。
\end{tcolorbox}




