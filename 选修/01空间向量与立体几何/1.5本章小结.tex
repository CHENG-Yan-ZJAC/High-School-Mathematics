\section{本章小结}

引入向量后,所有的几何问题都有了明显的思路,只是计算量大小的问题。任何几何问题采用暴力计算的方法一定能求解,但这显然不是高效的方法。要学好几何,则不可偏用向量,适当结合一定的纯几何可大大降低计算量。

%============================================================
\subsection{习题}

\begin{example}[拓广探索16,难度:$\star \star $]
如图,在棱长为$a$的正方体$OABC-O'A'B'C'$中,$E,F$分别是棱$AB,BC$上的动点,且$AE=BF$。
\begin{itemize}
    \item 求证:$A'F\bot C'E$;
    \item 当三棱锥$B'-BEF$的体积取得最大值时,求平面$B'EF$与平面$BEF$夹角的正切值。
\end{itemize}
\end{example}

\begin{figure}[h]
\centering
\begin{tikzpicture}[style={x={(-145:0.5)},y={(1cm,0)},z={(0,1cm)}}, line join=round, scale=2]
\mydrawcube[1]{O}{A}{B}{C}{O'}{A'}{B'}{C'}
\coordinate[label=below left: {$F$}] (F) at ($(B)!0.7!(C)$);
\coordinate[label=below:      {$E$}] (E) at ($(A)!0.7!(B)$);
\coordinate[label=above:      {$y$}] (y) at ($(C)!0.5!(F)$);
\coordinate[label=below right:{$x$}] (x) at ($(B)!0.5!(E)$);
\draw[dashed] (C')--(C) (O)--(C)--(B) (B')--(F)--(E);
\draw[thick] (B')--(E);
\draw[dashed,blue] (C')--(E) (A')--(F);
\fill[pink!70!white,opacity=0.5] (F)--(E)--(B)--cycle;
\fill[blue!50!white,opacity=0.5] (F)--(E)--(B')--cycle;
\end{tikzpicture}
\end{figure}

解:

(1)在$E,F$均是动点的前提下,显然不能用纯几何的方法求证。考虑采用向量的方法,以$O$为原点建立直角坐标系,于是:
\begin{align*}
&\because \begin{cases}
	F=\left( 0,y,0 \right) ,E=\left( x,a,0 \right)\\
	A'=\left( a,a,a \right) ,C'=\left( 0,0,a \right)\\
\end{cases} \\
&\therefore \begin{cases}
	\overrightarrow{A'F}=\left( -a,y-a,-a \right)\\
	\overrightarrow{C'E}=\left( x,a,-a \right)\\
\end{cases} \\
&\therefore \overrightarrow{A'F}\cdot \overrightarrow{C'E}=-ax+a\left( y-a \right) +a^2=-ax+ay \\
&\because AE=BF \\
&\therefore EB=CF \\
&\therefore x=y \\
&\therefore \overrightarrow{A'F}\cdot \overrightarrow{C'E}=0
\end{align*}

(2)三棱锥$B'-BEF$的体积
\[
V_{B'-BEF}=\frac{1}{3}\cdot S_{\bigtriangleup BEF}\cdot BB'
\]
显然当$S_{\bigtriangleup BEF}$最大时,$V_{B'-BEF}$最大,于是:
\begin{align*}
&\because S_{\bigtriangleup BEF}=\frac{1}{2}\cdot BF\cdot BE=\frac{1}{2}\cdot \left( a-y \right) \cdot x\\
&\because x=y \\
&\therefore S_{\bigtriangleup BEF}=\frac{1}{2}\left( -x^2+ax \right) \quad x\in \left( 0,a \right)
\end{align*}
显然当$x=y=1/2$时,$V_{B'-BEF}$最大,此时,$B'EF$和$BEF$均为等腰三角形,后略。

深入分析:

有兴趣可以计算一下$V_{E-FA'B'}$,提示$V_{E-FA'B'}=V_{B-FA'B'}$。再看一下$V_{E-FA'B'}$什么时候取得最大值。

\begin{tcolorbox}
该题是典型的通过向量联系几何和代数的题目,整体思路还是明显的,通过向量将几何问题转化为代数问题,剩下的就是函数了。
\end{tcolorbox}




