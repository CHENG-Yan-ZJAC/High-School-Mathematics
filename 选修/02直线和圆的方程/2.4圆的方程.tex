\section{圆的方程}

本节要点:
\begin{itemize}
    \item 掌握2种圆的代数形式;
    \item 深刻理解它们代表的几何意义。
\end{itemize}

~

{\bf 标准方程}:
\[
\left( x-a \right) ^2+\left( y-b \right) ^2=r^2
\]

{\bf 一般方程}:
\[
x^2+y^2+Dx+Ey+F=0
\]

\begin{tcolorbox}
详细阅读所有例题。例2和例4都是已知求三角形的外接圆,分别用标准方程和一般方程,体会两者的计算量。例5中$M$是中点,若$M$不是中点,而是某个等比例点,则$M$的运动轨迹是什么样子?再如果$A$是任意轨迹,则$M$的运动轨迹和$A$的轨迹有什么关系?早期钟表内部的细小零件的制作就是依据这个原理,XML。
\end{tcolorbox}

%============================================================
\subsection{拓展讨论:圆的定义}

我们给出圆的定义,然后推导圆的公式。

\begin{definition}[圆]
设二维平面中有一定点$P_0$,若平面上的点和$P_0$的距离都是$r>0$,则称这些点组成的集合为{\bf 圆},常用$C$表示,即:
\[
C:=\left\{ P \middle| \left| \overrightarrow{P_0P} \right|=r \right\}
\]
若令$P=\left( x,y \right) ,P_0=\left( x_0,y_0 \right) $,则圆可以表示为下列的{\bf 向量方程}:
\[
\left( x-x_0 \right) ^2+\left( y-y_0 \right) ^2=r^2
\]
使用复数的概念,可得到{\bf 参数方程}:
\[
\begin{cases}
	x-x_0=r\cos \theta\\
	y-y_0=r\sin \theta\\
\end{cases} \quad \theta \in \left[ 0,2\pi \right)
\]
\end{definition}

再考虑一个问题,考察课本给出的圆的一般方程
\[
x^2+y^2+Dx+Ey+F=0
\]
$A,B,C$去哪里了?

事实上,关于$x,y$的二次曲线,更一般的应该是:
\[
Ax^2+By^2+Cxy+Dx+Ey+F=0
\]
是一个倾斜的椭圆,当$C=0$时,椭圆归正,即长短轴平行于{it xy}轴,进一步地,当$A=B$时,为圆。

%============================================================
\subsection{习题}

\begin{example}[综合运用5,难度:$\star \star $]
已知圆的一条直径的端点分别是$A\left( x_1,y_1 \right) $,$B\left( x_2,y_2 \right) $,求证此圆的方程是
\[
\left( x-x_1 \right) \left( x-x_2 \right) +\left( y-y_1 \right) \left( y-y_2 \right) =0
\]
\end{example}

解:

用向量的方法,向量表示的点在圆上,则构成直角三角形,两个直角边的内积为0。

\begin{tcolorbox}
本题联合向量内积的几何意义,一句话就能证明,非常容易。
\end{tcolorbox}

~

\begin{example}[综合运用7,难度:$\star $]
已知等腰三角形$ABC$的一个顶点为$A\left( 4,2 \right) $,底边的一个端点为$B\left( 3,5 \right) $,求底边的另一个端点$C$的轨迹方程,并说明它是什么图形。
\end{example}

解:

显然$AB=AC$,但$A,B,C$不能构成直线,于是
\[
\left( x-4 \right) ^2+\left( y-2 \right) ^2=10 \qquad x\ne 3,5
\]

\begin{tcolorbox}
此题不难,但需注意$x$的取值。
\end{tcolorbox}




