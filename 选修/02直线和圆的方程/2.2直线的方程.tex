\section{直线的方程}

本节要点:
\begin{itemize}
    \item 掌握5种直线的代数形式;
    \item 深刻理解它们代表的几何意义。
\end{itemize}

~

{\bf 点斜式},$k$必须有定义,即不能垂直于{\it x}轴:
\[
y-y_0=k\left( x-x_0 \right)
\]

{\bf 斜截式},$k$必须有定义,即不能垂直于{\it x}轴:
\[
y=kx+b
\]

{\bf 两点式},对两点有要求$x_1\ne x_2$且$y_1\ne y_2$:
\[
\frac{y-y_1}{y_2-y_1}=\frac{x-x_1}{x_1-x_2}
\]

{\bf 截距式},任意一个截距都不能为0:
\[
\frac{x}{a}+\frac{y}{b}=1
\]

{\bf 一般式},任何情况都能用:
\[
Ax+By+C=0
\]

%============================================================
\subsection{拓展讨论:直线的定义}

其实我们可以参考上一章中对空间直线的描述,定义平面上的直线。

\begin{definition}[直线]
若二维平面中有一个给定点$P_0$和一个给定的向量$\boldsymbol{n}$,若平面上的点和$P_0$构成的向量与$\boldsymbol{n}$平行,则称这些点组成的集合为{\bf 直线},常用$l$表示,即:
\[
l:=\left\{ P \middle| \overrightarrow{P_0P}=\lambda \boldsymbol{n} \right\}
\]
若令$P=\left( x,y \right) ,P_0=\left( x_0,y_0 \right) ,n=A\boldsymbol{x}+B\boldsymbol{y}$,则直线可以表示为下列的{\bf 向量方程}:
\[
\left( x-x_0,y-y_0 \right) =\lambda \left( A,B \right)
\]
若保留$\lambda $,可得到{\bf 参数方程}:
\[
\begin{cases}
	x=\lambda A+x_0\\
	y=\lambda A+y_0\\
\end{cases} \quad \lambda \in \mathbb{R}
\]
若约去$\lambda $,则可得到一般式:
\begin{align*}
&\frac{x-x_0}{A}=\frac{y-y_0}{B} \\
&\left( -Bx \right) +\left( Ay \right) +\left( Bx_0-Ay_0 \right) =0
\end{align*}
\end{definition}

%============================================================
\subsection{习题}

\begin{example}[综合运用12,难度:$\star $]
若直线$l$沿{\it x}轴向左平移3个单位长度,再沿{\it y}轴向上平移1个单位长度后,回到原来的位置,试求直线$l$的斜率。
\end{example}

解一:

从定义出发,设直线$y-y_0=k\left( x-x_0 \right) $,有:
\begin{align*}
&\because y-y_0-1=k\left( x-x_0+3 \right) \\
&\therefore y-y_0-1=k\left( x-x_0 \right) +3k \\
&\therefore k=-\frac{1}{3}
\end{align*}

解二:

从向量的角度,设点直线上的点$\left( x_0,y_0 \right) $,平移后到达$\left( x_0-3,y_0+1 \right) $,又这两点都在直线上,于是:
\[
\frac{\left( y_0+1 \right) -y_0}{\left( x_0-3 \right) -x_0}=k
\]

\begin{tcolorbox}
本题考察斜率的定义。
\end{tcolorbox}




