\section{直线的倾斜角与斜率}

本节要点:
\begin{itemize}
    \item 掌握倾斜角的概念;
    \item 掌握斜率的概念。
\end{itemize}

%============================================================
\subsection{倾斜角和斜率}

\begin{definition}[倾斜角和斜率]
我们将直线$l$的向上的方向和{\it x}轴正向所成的角称为{\bf 倾斜角},通常用希腊字母表示,不难得到$\alpha \in \left[ 0,\pi \right) $。我们称倾斜角的正切值为该斜线的{\bf 斜率},通常记作$k$,即:
\[
k:=\tan \alpha
\]
\end{definition}

\begin{tcolorbox}
注意,垂直于{\it x}轴的直线是没有斜率的。
\end{tcolorbox}

\begin{tcolorbox}
想两个问题。首先,既然有了倾斜角,为什么还要定义斜率?其次,既然要定义斜率,为什么要用正切值?用正弦或余弦不行吗?何况正弦和余弦在$\alpha \in \left[ 0,\pi \right) $上都有定义,不会像正切一样存在不定义点。XML。
\end{tcolorbox}

%============================================================
\subsection{两条直线平行和垂直的判定}

\begin{theorem}
设直线$l_1,l_2$的斜率为$k_1,k_2$,则有:
\begin{align*}
&l_1\parallel l_2\Leftrightarrow k_1=k_2 \\
&l_1\bot l_2\Leftrightarrow k_1k_2=-1
\end{align*}
\end{theorem}

\begin{tcolorbox}
注意,各自平行于两个轴的直线是没有办法用上述定理判断垂直的。
\end{tcolorbox}




