\section{直线与圆、圆与圆的位置关系}

本节要点:
\begin{itemize}
    \item 从代数角度深刻理解线圆位置关系;
    \item 从代数角度深刻理解圆圆位置关系。
\end{itemize}

%============================================================
\subsection{直线与圆的位置关系}

我们可以根据线圆位置关系的定义总结出判定定理。

\begin{theorem}
设二维平面有直线$l$和圆$C$,将直线方程带入圆方程得到一个关于$x$或$y$的一元二次方程,则有:
\begin{itemize}
    \item $\varDelta >0\Leftrightarrow \text{直线和圆相交}$;
    \item $\varDelta =0\Leftrightarrow \text{直线和圆相切}$;
    \item $\varDelta <0\Leftrightarrow \text{直线和圆相离}$。
\end{itemize}
\end{theorem}

%============================================================
\subsection{圆与圆的位置关系}

同上,我们可以总结出判定定理。

\begin{theorem}
设二维平面有两个圆$C_1,C_2$,将两个圆方程合并后得到一个关于$x$或$y$的一元二次方程,则有:
\begin{itemize}
    \item $\varDelta >0\Leftrightarrow \text{圆和圆相交}$;
    \item $\varDelta =0\Leftrightarrow \text{圆和圆相切}$;
    \item $\varDelta <0\Leftrightarrow \text{圆和圆相离}$。
\end{itemize}
\end{theorem}

这里需要注意:
\begin{itemize}
    \item 两个圆的方程可能是一模一样,说明两个圆重合,该判定定理自然无法使用;
    \item 相切可能是外切,也可能是内切,需要结合圆心距离进一步判定;
    \item 相离也是有外部和内部两个可能,同样需要结合圆心距判定。
\end{itemize}

%============================================================
\subsection{拓展讨论:两个圆方程相减的图形}

设两圆$C_1,C_2$相交,将它们的一般方程相减必然得到一直线方程,也即两个交点确定一条直线,这点毋庸置疑,而且几何意义也十分清楚。

那么将两圆方程相加呢?由于二次项不能抵消,必然也是一个圆,用标准方程推导可以直观地看清圆心和半径。设两圆$C_1,C_2$如下:
\[
\begin{cases}
	\left( x-x_1 \right) ^2+\left( y-y_1 \right) ^2={r_1}^2\\
	\left( x-x_2 \right) ^2+\left( y-y_2 \right) ^2={r_2}^2\\
\end{cases}
\]
相加,化简后:
\[
\left( x-\frac{x_1+x_2}{2} \right) ^2+\left( y-\frac{y_1+y_2}{2} \right) ^2=\frac{{r_1}^2+{r_2}^2}{2}-\frac{\left( x_1-x_2 \right) ^2+\left( y_1-y_2 \right) ^2}{4}
\]
其中$\left( x_1-x_2 \right) ^2+\left( y_1-y_2 \right) ^2$为圆心距,所以也可以表示为:
\[
\left( x-\frac{x_1+x_2}{2} \right) ^2+\left( y-\frac{y_1+y_2}{2} \right) ^2=\frac{{r_1}^2+{r_2}^2}{2}-\frac{C_1{C_2}^2}{4}
\]
所以,两个圆方程相加的方程表示一个圆$C_3$,圆心和半径如下:
\begin{align*}
&\left( x_3,y_3 \right) =\left( \frac{x_1+x_2}{2},\frac{y_1+y_2}{2} \right) \\
&{r_3}^2=\frac{{r_1}^2+{r_2}^2}{2}-\frac{C_1{C_2}^2}{4} \qquad C_1C_2\in \left[ \left| r_1-r_2 \right|,r_1+r_2 \right]
\end{align*}
$r_3\in \left[ \frac{\left| r_1-r_2 \right|}{2},\frac{r_1+r_2}{2} \right] $,分别当两圆内切和外切时,取到区间两侧。圆心非常好理解,半径较难理解,我们从向量的角度出发研究。

如下图,圆$C_1,C_2$交于$M,N$,新圆的圆心为$C_3$:

\begin{figure}[h]
\centering
\begin{tikzpicture}[line join=round, scale=0.75]
\coordinate[label=below:{$C_1$}] (C1) at (0,0);
\coordinate[label=below:{$C_2$}] (C2) at (2,0);
\coordinate[label=below:{$C_3$}] (C3) at (1,0);
\draw[thick,name path=c1] (C1) circle (2);
\draw[thick,name path=c2] (C2) circle (1.5);
\path[name intersections={of=c1 and c2}]
    coordinate[label=above:{$M$}] (M) at (intersection-1)
    coordinate[label=below:{$N$}] (N) at (intersection-2);
\draw (C1)--(M)--(C2)--(C1) (M)--(C3);
\end{tikzpicture}
\end{figure}

\begin{align*}
&\because \left| \overrightarrow{MC_3} \right|=\left| \overrightarrow{MC_1}+\overrightarrow{MC_2} \right|/2 \\
&\therefore \left| \overrightarrow{MC_3} \right|^2=\frac{{r_1}^2+{r_2}^2+\overrightarrow{MC_1}\cdot \overrightarrow{MC_2}}{4} \\
&\because \left| \overrightarrow{C_1C_2} \right|=\left| \overrightarrow{MC_2}-\overrightarrow{MC_1} \right| \\
&\therefore \left| \overrightarrow{C_1C_2} \right|^2={r_1}^2+{r_2}^2-\overrightarrow{MC_1}\cdot \overrightarrow{MC_2} \\
&\therefore \left| \overrightarrow{MC_3} \right|^2=\frac{{r_1}^2+{r_2}^2+\left( {r_1}^2+{r_2}^2-\left| \overrightarrow{C_1C_2} \right|^2 \right)}{4}
\end{align*}




