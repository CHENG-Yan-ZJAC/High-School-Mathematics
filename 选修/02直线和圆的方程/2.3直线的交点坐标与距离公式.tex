\section{直线的交点坐标与距离公式}

本节要点:
\begin{itemize}
    \item 从二元一次方程组的解出发理解直线的几何关系;
    \item 熟练掌握点到直线的距离公式;
    \item 掌握平行线间的距离公式。
\end{itemize}

~

点线距离公式:
\[
d=\frac{\left| Ax_0+By_0+C \right|}{\sqrt{A^2+B^2}}
\]

线线距离公式:
\[
d=\frac{\left| C_1-C_2 \right|}{\sqrt{A^2+B^2}}
\]

\begin{tcolorbox}
仔细阅读距离公式的推导过程。
\end{tcolorbox}

%============================================================
\subsection{习题}

\begin{example}[综合运用14,难度:$\star $]
已知$A\left( -3,-4 \right) ,B\left( 6,3 \right) $两点到直线$l:ax+y+1=0$的距离相等,求$a$的值。
\end{example}

解一:

根据点线距离公式:
\begin{align*}
&\because d_A=\frac{\left| -3a-4+1 \right|}{\sqrt{a^2+1}} \\
&\because d_B=\frac{\left| Ax_0+By_0+C \right|}{\sqrt{A^2+B^2}}=\frac{\left| 6a+3+1 \right|}{\sqrt{a^2+1}} \\
&\because \left( -3a-3 \right) ^2=\left( 6a+4 \right) ^2 \\
&\therefore 27a^2+30a+7=0 \\
&\therefore a=-\frac{1}{3}\mathrm{or}-\frac{7}{9}
\end{align*}

解二:

根据几何关系,若两点不在同一侧,则线段$AB$的中点$\left( \frac{3}{2},-\frac{1}{2} \right) $在直线上,带入直线方程易得:
\[
a=-\frac{1}{3}
\]
或者两点在同侧,易得向量$\overrightarrow{AB}=\left( 9,7 \right) $和直线平行,于是:
\[
k=-a=\frac{7}{9}
\]

\begin{tcolorbox}
观察几何关系可以降低计算量。
\end{tcolorbox}

~

\begin{example}[拓广探索11,难度:$\star \star $]
已知$0<x<1,0<y<1$。
\begin{enumerate}
    \item 求证:$\sqrt{x^2+y^2}+\sqrt{x^2+\left( 1-y \right) ^2}+\sqrt{\left( 1-x \right) ^2+y^2}$
    
    $+\sqrt{\left( 1-x \right) ^2+\left( 1-y \right) ^2}\geqslant 2\sqrt{2}$,并求使等式成立的条件。
    \item 说明上述不等式的几何意义。
\end{enumerate}
\end{example}

解:

构建如下单位正方形$OABC$,$P$在正方形内部。

\begin{figure}[h]
\centering
\begin{tikzpicture}[line join=round, scale=2]
\mydrawxy{0}{1.3}{0}{1.2}
\mydrawsquare[1]{O}{A}{B}{C}
\draw[dashed] (A)--(C) (B)--(O);
\coordinate[label=left:{$P$}] (P) at (0.3,0.5);
\draw[dashed,blue] (C)--(P)--(A) (O)--(P)--(B);
\end{tikzpicture}
\end{figure}

根据三角形三边长度关系易得:
\begin{itemize}
    \item $\sqrt{x^2+y^2}+\sqrt{\left( 1-x \right) ^2+\left( 1-y \right) ^2}\geqslant \sqrt{2}$,当且仅当$P$在直线$OB$上时等号成立;
    \item $\sqrt{x^2+\left( 1-y \right) ^2}+\sqrt{\left( 1-x \right) ^2+y^2}\geqslant \sqrt{2}$,当且仅当$P$在直线$AC$上时等号成立;
\end{itemize}
所以,当且仅当$P$在$OB,AC$的交点时求证的等式的等号成立。

\begin{tcolorbox}
使用几何可以非常容易证明此题。
\end{tcolorbox}




