\section{本章小结}

本章用代数的方法讨论了直线和圆。注意,我们有了第一章的知识,学习时更要注重数形结合。学习和做题时,必须从几何出发,在理解几何关系的基础上运用代数的方法。

%============================================================
\subsection{习题}

\begin{example}[拓广探索20,难度:$\star $]
已知圆$C:\left( x-1 \right) ^2+\left( y-2 \right) ^2=25$,直线$l:\left( 2m+1 \right) x+\left( m+1 \right) y-7m-4=0$。
\begin{enumerate}
    \item 求证:直线$l$恒过定点。
    \item 直线$l$被圆$C$截得的弦何时最长、何时最短?并求截得的弦长最短时$m$的值以及最短弦长。
\end{enumerate}
\end{example}

解一:

(1)恒过定点,表示取$m_1\ne m_2$生成两条直线,交点必须和$m_1,m_2$无关,联立直线方程:
\[
\begin{cases}
	\left( 2m_1+1 \right) x+\left( m_1+1 \right) y-7m_1-4=0\\
	\left( 2m_2+1 \right) x+\left( m_2+1 \right) y-7m_2-4=0\\
\end{cases}
\]
消元得$\left( m_2-m_1 \right) y-\left( m_2-m_1 \right) =0$,可见只有当$y=1$时与$m_1,m_2$无关,带入直线方程得$\left( 2m+1 \right) x-6m-3=0$,易得恒过$\left( 3,1 \right) $。

(2)易得恒定点在园内,于是弦有最短且垂直于直径方向,否则园外点可以相切,最短为0,最长弦自然是直径。略。

\begin{figure}[h]
\centering
\begin{tikzpicture}[line join=round, scale=0.3]
\pgfmathparse{0.6/0.3}
\mydrawxy{0}{7}{0}{8}
\draw[thick] (1,2) circle(5);
\coordinate[label=left: {$C\left( 1,2 \right) $}] (C) at (1,2);
\coordinate[label=right:{$P\left( 3,1 \right) $}] (P) at (3,1);
\fill (C) circle (\pgfmathresult mm);
\fill (P) circle (\pgfmathresult mm);
\draw[dashed] ($(C)!-3.0!(P)$)--($(C)!3.0!(P)$) (0,-5)--(6,7);
\end{tikzpicture}
\end{figure}

解二:

(1)过恒定点的意思是有一点$\left( x_0,y_0 \right) $必然在直线上,而且不依赖于$m$的值,也就是说该点坐标的表达式中不能出现$m$,将直线写成:
\[
\left( 2x_0+y_0-7 \right) m+x_0+y_0-4=0
\]
要使之不出现$m$,则有:
\[
\begin{cases}
	2x_0+y_0-7=0\\
	x_0+y_0-4=0\\
\end{cases}
\]
可得$\left( x_0,y_0 \right) =\left( 3,1 \right) $。

\begin{tcolorbox}
理解“恒过一点”的几何含义,本题没有难度。
\end{tcolorbox}




