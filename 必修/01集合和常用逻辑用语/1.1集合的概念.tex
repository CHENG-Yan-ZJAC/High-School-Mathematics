\section{集合的概念}

本节要点:
\begin{itemize}
    \item 掌握集合的概念;
    \item 熟练掌握集合的描述法。
\end{itemize}

~

集合的概念并没有什么难度,重在理解元素。元素可以是数、点、或者任何可以准确描述的事物,但必须满足:
\begin{itemize}
    \item {\bf 确定性}:元素$a$要么在集合$A$中,要么不在$A$中,逻辑上必须清晰;
    \item {\bf 唯一性}:集合$A$中不能存在两个一模一样的元素,用数学语言表达就是$\forall a,b\in A,a\ne b$;
    \item {\bf 无序性}:集合$A$中的元素$a$和$b$,我们只讨论其存在,不关心它们的顺序。
\end{itemize}

对于集合的表示方法,教材写地有些凌乱。掌握一点,数学中,我们用花括号描述集合,如:
\[
A=\left\{ \text{这是一个集合} \right\}
\]
花括号中的描述方法没有特别的格式规范,只要求明确清晰即可。一般地,我们用竖线分割,前半部用小写字母表示元素,后半部用表达式确定元素的范围或关系等限定,如果有多个限定,则用逗号分隔,如下:
\begin{align*}
&A=\left\{ x \middle| x>3 \right\} \\
&B=\left\{ \left( x,y \right) \middle| x^2+y^2>1 \right\} \\
&C=\left\{ x \middle| x\in \mathbb{R} ,x^2=-1 \right\}
\end{align*}
集合$A$表示了数轴上的一个段,其元素是单个数$x$;集合$B$表示了平面上的一个圆,其元素是平面上的点,用一个有序数对$\left( x,y \right) $表示;集合$C$描述了一个空集,和$C=\oslash $是一样的,这说明集合的描述并不唯一。

\begin{tcolorbox}
不必刻意理解教材中的列举法、描述法等方法,后面到了函数阶段,定义域、值域都是采用这种描述方法,所以只需理解并熟练掌握上述描述方法即可。
\end{tcolorbox}

\begin{tcolorbox}
广义来讲,集合和元素的概念非常宽泛,没有限定必须是数字。如我们可以将高一(2)班作为一个集合,那么班级里的同学可以视作元素。我们也可以将所有多项式组成一个集合。我们更可以将马达加斯加的雄性企鹅组成一个集合。
\end{tcolorbox}




