\section{充分条件和必要条件}

本节要点:
\begin{itemize}
    \item 了解命题的概念;
    \item 熟练掌握并深刻理解充分条件和必要条件的概念。
\end{itemize}

~

命题是一个陈述句,往往很简单,而且陈述是明确的、无歧义的,命题的结论要么为真、要么为假,只有其中一个结果,而且结果也是明确的、无歧义的。

对于充分条件和必要条件,下列四种说法等价:
\begin{itemize}
    \item 命题“若$p$,则$q$”为真命题;
    \item $p\Rightarrow q$;
    \item $p$是$q$的充分条件;
    \item $q$是$p$的必要条件。
\end{itemize}

\begin{tcolorbox}
对于命题、充分条件、必要条件,我们无需从形式逻辑方面完全掌握其概念,毕竟不是逻辑课程。我们只需了解它们是如何构建数学中的定义、定理、性质的。
\end{tcolorbox}

一般来讲,定义(definition)$D$是逻辑的起点,是一个人为的设定,不容置疑的。而判断一事物是否为$D$会比较复杂,定理(theorem)$T$给出了判断事物是否为$D$的简单方法,通常定理以充分条件的形式给出。一旦我们获得了一事物是$D$这个判断,那么该事物必然有$D$的性质(property)$P$,通常这类性质往往以必要条件的形式给出。
\[
T\rightarrow D\rightarrow P
\]

具体到解题中,已知条件通常以定理的前提给出,问题则以性质的结论要求计算或证明。所以,可以总结解题套路:
\begin{enumerate}
    \item 用定理将式子或图形框在某一定义中,如证明两个三角形相似即将两个三角形框在了相似这个定义中;
    \item 用性质得出结论,如计算另一个三角形的某个角。
\end{enumerate}
\[
\text{已知}\overset{T}{\Rightarrow}D\overset{P}{\Rightarrow}\text{结论}
\]

\begin{tcolorbox}
充要条件是整个高中乃至整个数学的思维范式!需要深刻领会。如何研究事物的运动规律?这是有一套固定的方法论的。数学上,我们不会单独讨论某个问题,而是将所有类似的问题总结提炼,形成定义、定理、性质。碰到具体问题,首先将问题转到数学空间,即归类到某个数学定义,然后用那个定义配套的定理和性质分析问题。这就是西方科学的基石,从亚里士多德开始的演绎法!
\end{tcolorbox}

\begin{tcolorbox}
演绎法是由一般到特殊的推理方法,与“归纳法”相对。最著名的演绎论恐怕莫过于:所有人都会死,苏格拉底是人,所以苏格拉底会死。演绎法中,“一般”是已知的事实,“特殊”是未知的结论。数学中,这个一般的已知的事实就是我们学习的各种数学概念、定理、公式,这个特殊就是各种题目。题目千千万万,但万变不离其宗,再抽象晦涩的题目总能化为最最基础的数学概念。事实上,高中数学也就是那么几个一个手数地过来的基础概念。
\end{tcolorbox}




