\section{集合的基本运算}

本节要点:
\begin{itemize}
    \item 熟练掌握集合的运算法则。
\end{itemize}

~

集合的运算并没有什么难度。教材介绍了3中运算,交并补,我们可以做如下精确的定义。

\begin{definition}[集合的运算]
设三个集合$A,B,C$,若$A$中的任一元素同时属于$C$,且$B$中的任一元素也同时属于$C$,则称{\bf $C$为$A,B$的并集},记作$A\cup B$,即:
\[
A\cup B:=\left\{ a \middle| a\in A\text{或}a\in B \right\}
\]
若$C$中的任一元素同时属于$A$和$B$,则称{\bf $C$为$A,B$的交集},记作$A\cap B$,即:
\[
A\cap B:=\left\{ a \middle| a\in A\text{且}a\in B \right\}
\]
若我们将$A$中的同时属于$B$的元素去除,得到的集合称为{\bf $A$和$B$的差},也称为{\bf $B$关于$A$的补集},记作$A\cap \bar{B}$,即:
\[
A\cap \bar{B}:=\left\{ a \middle| a\in A\text{且}a\notin B \right\}
\]
\end{definition}

\begin{figure}[h]
\centering
\begin{tikzpicture}[line join=round, scale=0.5]
% A in B
\coordinate (O) at (-5,1.6);
\draw[thick] ($(O)+(-2,-1)$)--($(O)+(2,-1)$)--($(O)+(2,1)$)--($(O)+(-2,1)$)--($(O)+(-2,-1)$);
\coordinate[label=above:{$A\subset B$}] (t) at ($($(O)+(-2,1)$)!0.5!($(O)+(2,1)$)$);
\draw[thick] ($(O)+(0,0)$) ellipse (1.5 and 0.8);
\draw[thick] ($(O)+(0,0)$) circle (0.5);
\coordinate[label=center:{$A$}] (a) at ($(O)+(0,0)$);
\coordinate[label=center:{$B$}] (b) at ($(O)+(1,0)$);
% A or B
\coordinate (O) at (0,1.6);
\draw[thick] ($(O)+(-2,-1)$)--($(O)+(2,-1)$)--($(O)+(2,1)$)--($(O)+(-2,1)$)--($(O)+(-2,-1)$);
\coordinate[label=above:{$A\cup B$}] (t) at ($($(O)+(-2,1)$)!0.5!($(O)+(2,1)$)$);
\fill[black!30!white] ($(O)+(-0.5,0)$) circle (0.8);
\fill[black!30!white] ($(O)+(0.5,0)$)  circle (0.8);
\draw[thick] ($(O)+(-0.5,0)$) circle (0.8);
\draw[thick] ($(O)+(0.5,0)$)  circle (0.8);
\coordinate[label=center:{$A$}] (a) at ($(O)+(-0.7,0)$);
\coordinate[label=center:{$B$}] (b) at ($(O)+(0.7,0)$);
% A and B
\coordinate (O) at (5,1.6);
\draw[thick] ($(O)+(-2,-1)$)--($(O)+(2,-1)$)--($(O)+(2,1)$)--($(O)+(-2,1)$)--($(O)+(-2,-1)$);
\coordinate[label=above:{$A\cap B$}] (t) at ($($(O)+(-2,1)$)!0.5!($(O)+(2,1)$)$);
\begin{scope}
\clip[draw] ($(O)+(-0.5,0)$) circle (0.8);
\fill[black!30!white] ($(O)+(0.5,0)$) circle (0.8);
\end{scope}
\draw[thick] ($(O)+(-0.5,0)$) circle (0.8);
\draw[thick] ($(O)+(0.5,0)$)  circle (0.8);
\coordinate[label=center:{$A$}] (a) at ($(O)+(-0.7,0)$);
\coordinate[label=center:{$B$}] (b) at ($(O)+(0.7,0)$);
% A - B
\coordinate (O) at (-5,-1.6);
\draw[thick] ($(O)+(-2,-1)$)--($(O)+(2,-1)$)--($(O)+(2,1)$)--($(O)+(-2,1)$)--($(O)+(-2,-1)$);
\coordinate[label=above:{$A-B=A\cap \bar{B}$}] (t) at ($($(O)+(-2,1)$)!0.5!($(O)+(2,1)$)$);
\fill[black!30!white] ($(O)+(-0.5,0)$) circle (0.8);
\draw[thick,fill=white] ($(O)+(0.5,0)$) circle (0.8);
\draw[thick] ($(O)+(-0.5,0)$) circle (0.8);
\coordinate[label=center:{$A$}] (a) at ($(O)+(-0.7,0)$);
\coordinate[label=center:{$B$}] (b) at ($(O)+(0.7,0)$);
% A and B is null
\coordinate (O) at (0,-1.6);
\draw[thick] ($(O)+(-2,-1)$)--($(O)+(2,-1)$)--($(O)+(2,1)$)--($(O)+(-2,1)$)--($(O)+(-2,-1)$);
\coordinate[label=above:{$A\cap B=\oslash $}] (t) at ($($(O)+(-2,1)$)!0.5!($(O)+(2,1)$)$);
\draw[thick] ($(O)+(-0.9,0)$) circle (0.8);
\draw[thick] ($(O)+(0.9,0)$)  circle (0.8);
\coordinate[label=center:{$A$}] (a) at ($(O)+(-0.9,0)$);
\coordinate[label=center:{$B$}] (b) at ($(O)+(0.9,0)$);
% not A
\coordinate (O) at (5,-1.6);
\fill[black!30!white] ($(O)+(-2,-1)$)--($(O)+(2,-1)$)--($(O)+(2,1)$)--($(O)+(-2,1)$)--cycle;
\draw[thick] ($(O)+(-2,-1)$)--($(O)+(2,-1)$)--($(O)+(2,1)$)--($(O)+(-2,1)$)--($(O)+(-2,-1)$);
\coordinate[label=above:{$\bar{A}$}] (t) at ($($(O)+(-2,1)$)!0.5!($(O)+(2,1)$)$);
\draw[thick,fill=white] ($(O)+(0,0)$) circle (0.8);
\coordinate[label=center:{$A$}] (a) at ($(O)+(0,0)$);
\end{tikzpicture}
\end{figure}

集合的运算除逻辑符号外,还可使用算术符号描述。

\begin{table}[h]
\centering
\begin{tabular}{ccc}
    \toprule
    运算 & 逻辑符号 & 算术符号\\
    \midrule
    并 & $A\cup B$ & $A+B$\\
    交 & $A\cap B$ & $AB$\\
    差 & $A\cap \bar{B}$ & $A-B$\\
    \bottomrule
\end{tabular}
\end{table}

\begin{tcolorbox}
集合的运算使得我们可以用一些简单集合的组合描述一个复杂集合。但要注意,这里的运算是逻辑运算,而非算术运算。
\end{tcolorbox}

集合的运算法则:
\begin{itemize}
    \item {\bf 交换律}:$AB=BA$
    \item {\bf 结合律}:$\left( A+B \right) +C=A+\left( B+C \right) ,\left( AB \right) C=A\left( BC \right) $
    \item {\bf 分配律}:$\left( A+B \right) C=\left( AC \right) +\left( AB \right) ,\left( AB \right) +C=\left( A+C \right) \left( B+C \right) $
    \item {\bf 德摩根律}:$\overline{A+B}=\bar{A}\bar{B},\overline{AB}=\bar{A}+\bar{B}$
\end{itemize}

~

其他常用运算公式:
\begin{align*}
& AB\subset A\subset A+B,\qquad AB\subset B\subset A+B \\
& AA=A \\
& A+A=A \\
& A+\bar{A}=\varOmega \\
& A-B=A\bar{B}=A-AB \\
& \left( A-B \right) +B=A+B
\end{align*}

\begin{tcolorbox}
上述运算法则和公式有些超纲,有兴趣可XML。
\end{tcolorbox}




