\section{事件的相互独立性}

本节要点:
\begin{itemize}
    \item 了解独立性的概念。
\end{itemize}

~

独立性是一个很深刻的概念,也是一个非常强的约束条件,反映了事件$A$的发生与否对事件$B$没有影响。其次要注意的是,我们一般不从定义判断两个事件是否独立,而是从事实角度判断,如果两个事件独立,则使用公式
\[
P\left( AB \right) =P\left( A \right) P\left( B \right)
\]
计算。通常而言,如果事件$A$描述了试验$E_A$的结果,事件$B$描述了试验$E_B$的结果,试验$E$由$E_A,E_B$构成,且$E_A,E_B$不存在相互逻辑关系,则$A,B$相互独立。




