\section{复数的四则运算}

本节要点:
\begin{itemize}
    \item 掌握复数的运算法则;
    \item 理解复数运算的几何意义。
\end{itemize}

~

复数的加减和向量加减一样,但乘除和向量完全不同。
\begin{align*}
&z_1\pm z_2=\left( a_1\pm a_2 \right) +\left( b_1\pm b_2 \right) i \\
&z_1\cdot z_2=\left( a_1+b_1i \right) \cdot \left( a_2+b_2i \right) =\left( a_1a_2-b_1b_2 \right) +\left( a_1b_2+a_2b_1 \right) i
\end{align*}

加法的交换律和结合律:
\begin{align*}
&z_1+z_2=z_2+z_1 \\
&\left( z_1+z_2 \right) +z_3=z_1+\left( z_2+z_3 \right)
\end{align*}

数乘的结合律和分配律:
\begin{align*}
&\lambda \left( \mu z \right) =\left( \lambda \mu \right) z \\
&\left( \lambda +\mu \right) z=\lambda z+\mu z \\
&\lambda \left( z_1+z_2 \right) =\lambda z_1+\lambda z_2
\end{align*}

乘法的交换律、结合律和分配律:
\begin{align*}
&z_1z_2=z_2z_1 \\
&\left( z_1z_2 \right) z_3=z_1\left( z_2z_3 \right) \\
&z_1\left( z_2+z_3 \right) =z_1z_2+z_1z_3
\end{align*}

一个重要的不等式和一个重要的等式:
\begin{itemize}
    \item $\left| z_1+z_2 \right|\leqslant \left| z_1 \right|+\left| z_2 \right|$,当且仅当$z_1,z_2$重叠时等号成立;
    \item $\left| z_1\cdot z_2 \right|=\left| z_1 \right|\cdot \left| z_2 \right|$。
\end{itemize}
注意和向量的区别。




