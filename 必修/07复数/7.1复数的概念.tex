\section{复数的概念}

本节要点:
\begin{itemize}
    \item 掌握复数的概念;
    \item 熟悉复数的几何意义;
    \item 理解复数和向量的关系。
\end{itemize}

~

\[
z=a+bi
\]

复数的定义没啥难点,数学家已经帮我们起头了。复数的发明XML。

复数的几何意义对应一个二维平面,但特别注意,由于复数的乘法和向量的乘法(内积)定义不同,所以复数和向量不是同构的,因此,我们称复数的几何表示为复平面,称向量的几何表示为二维平面!XML。

学习到这里一定要注意区别复数和向量。

\begin{tcolorbox}
复数的物理意义需要在解微分方程时讨论,超出高中数学的范围,所以高中阶段不讨论复数的物理意义。我们需要知道,复数确实有物理意义,很多物理现象都涉及复数,XML。
\end{tcolorbox}

~

\begin{example}[拓广探索11,难度:$\star $]
在复平面内指出与复数$z_1=1+2i,z_2=\sqrt{2}+\sqrt{3}i,z_3=\sqrt{3}-\sqrt{2}i,z_4=-2+i$对应的点$z_1,z_2,z_3,z_4$,判断这4个点是否在同一个圆上,并证明你的结论。
\end{example}

解:

在同一个圆上,且圆心为原点。
\[
\left| z_1 \right|=\left| z_2 \right|=\left| z_3 \right|=\left| z_4 \right|=\sqrt{5}
\]

\begin{tcolorbox}
此题非常放水,圆心在原点,直接判断模,非常简单。
\end{tcolorbox}




