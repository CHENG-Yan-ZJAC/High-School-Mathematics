\section{函数\texorpdfstring{$y=A\sin \left( \omega x+\varphi \right) $}\ }

本节要点:
\begin{itemize}
    \item 结合图形理解函数的各个部分。
\end{itemize}

~

本章前5节都把三角函数作为公式看待,本节开始将三角函数作为函数考察。
\[
y=A\sin \left( \omega x+\varphi \right)
\]

我们从最简单的形式开始
\[
y=\sin x
\]
这里,引申出一个概念,我们将$x$称为{\bf 相位}(phase)。所谓的相位,就是相的位置,所谓的相,就是$y$,$y$会随着$x$的不同位置呈现出不同的样子。

我们将图象进行沿{\it x}轴平移,得到
\[
y=\sin \left( x+\varphi \right)
\]
这里,又引申出一个概念,我们将$\varphi $称为{\bf 初相},即初始相位。

第三步,我们将图象进行拉伸,得到
\[
y=\sin \left( \omega x+\varphi \right)
\]
这里,再引申出一个概念,由于三角函数的周期性,有:
\[
\omega \left( x+\frac{2\pi}{\omega} \right) +\varphi =\omega x+2\pi +\varphi
\]
也即函数的周期为$T=2\pi /\omega $,我们将其倒数称为角频$f=\frac{\omega}{2\pi}$。

最后,将图象沿{\it y}轴拉伸,得到
\[
y=A\sin \left( \omega x+\varphi \right)
\]
我们称$A$为振幅。




