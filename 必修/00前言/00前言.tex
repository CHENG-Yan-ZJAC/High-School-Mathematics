\chapter{前言}

数学必修,上下两册,讲了3大部分内容:
\begin{itemize}
    \item 思维范式:介绍了数学的思维范式——“定理—定义—性质”;
    \item 代数:介绍函数,扩展了数域;
    \item 几何:从纯几何角度介绍立体几何。
\end{itemize}

~

章节编排:
\begin{itemize}
    \item {\bf 第一章\ 集合和常用逻辑用语},35页:介绍了数学的思维范式,需要深刻理解体会;
    \item {\bf 第二章\ 不等式},23页:介绍了高中阶段用到的两个不等式,基本不等式和二次方程的最值,但凡涉及取值范围的,都逃不开本章的内容;
    \item {\bf 第三章\ 函数的概念和性质},44页:从本章开始介绍3类基本初等函数,本章介绍第一类基本初等函数幂函数;
    \item {\bf 第四章\ 指数函数与对数函数},59页:介绍第二类基本初等函数,指数函数和对数函数;
    \item {\bf 第五章\ 三角函数},91页:介绍第三类基本初等函数,三角函数,至本章,所有基本初等函数介绍完毕;
    \item {\bf 第六章\ 平面向量及其应用},62页:扩展数域至向量,使用代数的方法讨论平面几何问题;
    \item {\bf 第七章\ 复数},30页:扩展数域至复数,至此,代数部分介绍完毕;
    \item {\bf 第八章\ 立体几何初步},75页:简单几何体的表面积和体积,从纯几何角度讨论直线与平面的关系,至此,几何部分介绍完毕;
    \item {\bf 第九章\ 统计},55页:介绍数理统计的基础概念;
    \item {\bf 第九章\ 概率},42页:介绍概率论的基础概念,由于缺乏高等数学的基础,统计和概率两章都不会深入讨论。
\end{itemize}

~

本笔记选取有价值的课后习题,给出求解和难度评价。

难度评价标准:
\begin{itemize}
    \item $\star $:简单,考察基本定义等概念,解题方法有章可循;
    \item $\star \star $:中等,考察“定理—定义—性质”的数学范式,解题方法依然有章可循,可能多绕几个弯,类似“定理—定义1—性质—定义2—性质……”;
    \item $\star \star \star $:较难,需融汇数个知识点,要求对这几个知识点在方法论层面融会贯通。
    \item $\star \star \star \star $:最难,考察实际问题,并没有直接给出明确的数学问题,需自行建模,所以需要我们在方法论和哲学层面融会贯通所有知识点。
\end{itemize}




