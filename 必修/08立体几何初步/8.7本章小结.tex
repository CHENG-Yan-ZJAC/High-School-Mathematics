\section{本章小结}

本章的重点在线面关系,难点在于垂直关系。线面关系是面面关系的基础,而线线关系又是线面关系的基础。{\bf 所以一切空间关系,归根结底还是平面关系。}解题时需要紧紧抓住这一点,将空间中的点线元素通过平移、作垂线等方法放到同一个平面中讨论。

我们在平面几何中详细讨论了点线面的关系,总结出了相当完善的理论,小到直线平行和垂直,大到三角形相似,甚至有平面向量。所以,要善用已知考察未知,用平面考察立体,牢牢记住{一切空间关系,归根结底还是平面关系。}

%============================================================
\subsection{习题}

\begin{example}[综合运用11,难度:$\star \star \star $]
如下图,在四面体$A-BCD$中,$M$是$AD$的中点,$P$是$BM$的中点,点$Q$在线段$AC$上,且$AQ=3QC$。求证:$PQ\parallel \text{平面}BCD$。
\end{example}

\begin{figure}[h]
\centering
\begin{minipage}{.49\textwidth}
\centering
\begin{tikzpicture}[style={x={(-145:0.5)},y={(1cm,0)},z={(0,1cm)}}, line join=round, scale=1.5]
\coordinate[label=above:{$A$}] (A) at (1,1,1.5);
\coordinate[label=left: {$B$}] (B) at (0,-1,0);
\coordinate[label=below:{$C$}] (C) at (2,1,0);
\coordinate[label=right:{$D$}] (D) at (0,1,0);
\coordinate[label=right:{$M$}] (M) at ($(A)!0.5!(D)$);
\coordinate[label=above:{$P$}] (P) at ($(B)!0.5!(M)$);
\coordinate[label=right:{$Q$}] (Q) at ($(A)!0.75!(C)$);
\coordinate[label=below:{$E$}] (E) at ($(A)!1.33!(P)$);
\draw[thick] (A)--(B)--(C)--(A)--(D)--(C);
\draw[dashed] (A)--(E)--(C) (M)--(B)--(D);
\draw[dashed,blue] (P)--(Q);
\fill[blue!50!white,opacity=0.5] (B)--(C)--(D)--cycle;
\end{tikzpicture}
\end{minipage}
\begin{minipage}{.49\textwidth}
\centering
\begin{tikzpicture}[line join=round, scale=1.5]
\coordinate[label=above:{$A$}] (A) at (0.5,1.5);
\coordinate[label=left: {$B$}] (B) at (-1,0);
\coordinate[label=right:{$D$}] (D) at (1,0);
\coordinate[label=right:{$M$}] (M) at ($(A)!0.5!(D)$);
\coordinate[label=above:{$P$}] (P) at ($(B)!0.5!(M)$);
\coordinate[label=below:{$E$}] (E) at ($(A)!1.33!(P)$);
\draw[thick] (A)--(B)--(D)--(A);
\draw[dashed] (A)--(E) (M)--(B);
\end{tikzpicture}
\end{minipage}
\end{figure}

解:

证明线面平行,就是证明线线平行,即在$BCD$中寻找一条线和$PQ$共线,很自然作$AP$并延长交$DB$于$E$。于是只需证明$PQ\parallel EC$即可,由于$AQ=3QC$,也即只需证明$AP=3PE$。将$ABD$单独拉出来,如上右图。最直观的就是使用平面向量,设$A=\left( a,b \right) ,D=\left( c,0 \right) $,则有:
\[
M=\left( \frac{a+c}{2},\frac{b}{2} \right) ,P=\left( \frac{a+c}{4},\frac{b}{4} \right)
\]
可构建直线$AE$的方程,求解得到$E$的坐标:
\[
E=\left( \frac{c}{3},0 \right)
\]
不难发现$\left| \overrightarrow{EA} \right|=4\left| \overrightarrow{EP} \right| $,余下略,证毕。

\begin{tcolorbox}
本题有$AP=3PE$之类不伦不类的数量关系,化到平面使用向量方法是最直观的。
\end{tcolorbox}

~

\begin{example}[综合运用12,难度:$\star \star $]
如图,在正方体$ABCD-A_1B_1C_1D_1$中,求证:
\begin{enumerate}
    \item $B_1D\bot \text{平面}A_1BC_1$;
    \item $B_1D$与平面$A_1BC_1$的交点$H$是$\bigtriangleup A_1BC_1$的重心。
\end{enumerate}
\end{example}

\begin{figure}[h]
\centering
\begin{tikzpicture}[style={x={(-145:0.5)},y={(1cm,0)},z={(0,1cm)}}, line join=round, scale=2]
\pgfmathparse{0.6/2.5}
\mydrawcube[1]{A}{B}{C}{D}{A_1}{B_1}{C_1}{D_1}
\coordinate[label=right:{$E$}] (E) at ($(B)!0.5!(C_1)$);
\draw[thick,blue] (A_1)--(C_1)--(B)--(A_1);
\draw[dashed,blue,name path=l1] (D)--(B_1);
\draw[dashed,red,name path=l2] (A_1)--(E);
\path [name intersections={of=l1 and l2}] coordinate[label=left:$H$] (H) at (intersection-1);
\fill (H) circle (\pgfmathresult mm);
\fill[blue!50!white,opacity=0.5] (A_1)--(C_1)--(B)--cycle;
\end{tikzpicture}
\end{figure}

解:

(1)还是较为简单,大致思路如下。

只需证明$B_1D\bot A_1B$且$B_1D\bot C_1B$且$B_1D\bot A_1C_1$,由于对称性,证明一个即可,可选$B_1D\bot A_1B$。不难证明$A_1B$平行于$B_1D$所在平面$ADC_1B_1$,证毕。

(2)连接$A_1,H$并延长交于$E$,不难发现$\bigtriangleup A_1BC_1$是等边三角形,也即证明$E$平分$BC_1$或$A_1E\bot BC_1$或$\angle C_1A_1E=\angle BA_1E$,观察后发现证明$A_1E\bot BC_1$较为简单。
\begin{align*}
&\because BC_1\bot A_1D \\
&\because BC_1\bot A_1B_1 \\
&\therefore BC_1\bot A_1DEB_1 \\
&\therefore BC_1\bot A_1E
\end{align*}

\begin{tcolorbox}
本题略有复杂,也是典型的空间几何证明题。大致套路就是若要证明线线垂直就先证明线面垂直,要证明线面垂直就先证明线线垂直。
\end{tcolorbox}

~

\begin{example}[综合运用13,难度:$\star $]
如图,在三棱锥$P-ABC$中,$\angle ACB=90^\circ$,$PA\bot \text{底面}ABC$。
\begin{enumerate}
    \item 求证:$\text{平面}PAC\bot \text{平面}PBC$;
    \item 若$AC=BC=PA$,$M$是$PB$的中点,求$AM$与平面$PBC$所成角的正切值。
\end{enumerate}
\end{example}

\begin{figure}[h]
\centering
\begin{tikzpicture}[style={x={(-145:0.5)},y={(1cm,0)},z={(0,1cm)}}, line join=round, scale=2]
\coordinate[label=above:       {$P$}] (P) at (0,0,1);
\coordinate[label=left:        {$A$}] (A) at (0,0,0);
\coordinate[label=right:       {$B$}] (B) at (0,1.414,0);
\coordinate[label=below:       {$C$}] (C) at (0.707,0.707,0);
\coordinate[label=above right: {$M$}] (M) at ($(P)!0.5!(B)$);
\coordinate[label=above left:  {$N$}] (N) at ($(P)!0.5!(C)$);
\draw[thick] (P)--(A)--(C)--(B)--(P)--(C) (A)--(N)--(M);
\draw[dashed] (A)--(B);
\draw[dashed,blue] (M)--(A);
\fill[blue!50!white,opacity=0.5] (P)--(C)--(B)--cycle;
\end{tikzpicture}
\end{figure}

解:

(1)略。

(2)关键在于找角,也即找到$A$在平面$PCB$上的垂足。从(1)已知$BC\bot PAC$,所以尽量考虑在$PC$上找一点,设$N$,使得$AN\bot PC$,于是$AN\bot PCB$,也即$N$就是垂足,题目要求的所成角即为$\angle NMA$。不难发现$\bigtriangleup PAC$为等腰直角三角形,于是易得:
\[
\tan \angle NMA=\frac{AN}{NM}=\frac{\sqrt{2}/2}{1/2}=\sqrt{2}
\]

\begin{tcolorbox}
本题关键在于找直线与平面的成角,没有其他方法,只有根据定义找。
\end{tcolorbox}

~

\begin{example}[综合运用14,难度:$\star $]
如图,在四棱锥$P-ABCD$中,底面$ABCD$为正方形,侧面$PAD$是正三角形,$\text{侧面}PAD\bot \text{底面}ABCD$,$M$是$PD$的中点。
\begin{enumerate}
    \item 求证:$AM\bot \text{平面}PCD$;
    \item 求侧面$PBC$与底面$ABCD$所成二面角的余弦值。
\end{enumerate}
\end{example}

\begin{figure}[h]
\centering
\begin{tikzpicture}[style={x={(-35:0.5)},y={(1cm,0)},z={(0,1cm)}}, line join=round, scale=1.25]
\coordinate[label=above:      {$P$}] (P) at (0,0,1.732);
\coordinate[label=left:       {$A$}] (A) at (0,-1,0);
\coordinate[label=below left: {$B$}] (B) at (2,-1,0);
\coordinate[label=below right:{$C$}] (C) at (2,1,0);
\coordinate[label=right:      {$D$}] (D) at (0,1,0);
\coordinate[label=right:      {$M$}] (M) at ($(P)!0.5!(D)$);
\coordinate[label=above left: {$N$}] (N) at ($(A)!0.5!(D)$);
\coordinate[label=below:      {$Q$}] (Q) at ($(B)!0.5!(C)$);
\draw[thick] (P)--(A)--(B)--(P)--(C)--(B);
\draw[dashed] (P)--(D)--(A) (C)--(D);
\draw[dashed,blue] (A)--(M);
\draw[thick,red] (P)--(Q);
\draw[dashed,red] (P)--(N)--(Q);
\fill[pink!70!white,opacity=0.5] (A)--(D)--(C)--(B)--cycle;
\fill[blue!50!white,opacity=0.5] (P)--(C)--(B)--cycle;
\end{tikzpicture}
\end{figure}

解:

(1)略。

(2)关键在于找二面角,也即在$BC$找一点$Q$,能够方便地作垂线。显然用$P$找$Q$较为方便,不难发现$\bigtriangleup PBC$为等腰三角形,于是很自然$Q$取$BC$中点,再取$AD$中点$N$。不难证明,$\angle NQP$就是要求的二面角,而且$\angle PNQ$是直角,于是:
\[
\cos \angle NQP=\frac{NQ}{PQ}=\frac{1}{\sqrt{7}/2}
\]

\begin{tcolorbox}
本题关键在于找二面角,同上题没有其他方法,只有根据定义找。
\end{tcolorbox}




