\section{用样本估计总体}

本节要点:
\begin{itemize}
    \item 了解概率分布;
    \item 掌握百分位数的概念;
    \item 掌握方差的概念。
\end{itemize}

~

本节介绍了数理统计中的剩余两个重要概念:百分位数(代表性的是中位数)和方差。学习是特别要注意它们的物理意义,XML。

\begin{tcolorbox}
教材中有一处错误。一般来讲,样本的方差应该是
\[
S^2=\frac{\sum_{i=1}^n{\left( X_i-\bar{X} \right)}^2}{n-1}
\]
分母是$n-1$,因为只有这样$S^2$才能是总体方差的无偏估计。而$\frac{1}{n}\sum_{i=1}^n{\left( X_i-\bar{X} \right)}^2$称为样本的2阶中心距。
\end{tcolorbox}




