\section{平面向量的应用}

本节要点:
\begin{itemize}
    \item 熟练掌握余弦定理;
    \item 熟练掌握正弦定理。
\end{itemize}

\begin{align*}
&\cos A=\frac{b^2+c^2-a^2}{2bc} \\
&\frac{a}{\sin A}=\frac{b}{\sin B}=\frac{c}{\sin C}
\end{align*}

~

\begin{example}[复习巩固3,难度:$\star $]
用向量法证明:直径所对的圆周角是直角。
\end{example}

解:

如下设置各变量。

\begin{figure}[h]
\centering
\begin{tikzpicture}[line join=round, scale=1.25]
\mydrawxy{-1.2}{2}{-1.2}{1.2}
\draw[thick] (0,0) circle(1);
\coordinate                                              (O) at (0,0);
\coordinate[label=above:      {$A\left( x,y \right) $}]  (A) at (-0.67,0.74);
\coordinate[label=left:       {$B\left( -r,0 \right) $}] (B) at (-1,0);
\coordinate[label=right:      {$C\left( r,0 \right) $}]  (C) at (1,0);
\coordinate[label=above left: {$\boldsymbol{c}$}]        (c) at ($(A)!0.5!(B)$);
\coordinate[label=above right:{$\boldsymbol{b}$}]        (b) at ($(A)!0.5!(C)$);
\draw[thick,-stealth] (A)--(B);
\draw[thick,-stealth] (A)--(C);
\draw[thick,-stealth] (B)--(C);
\end{tikzpicture}
\end{figure}

\begin{align*}
&\because \begin{cases}
	\boldsymbol{c}=\left( -r-x,-y \right)\\
	\boldsymbol{b}=\left( r-x,-y \right)\\
	x^2+y^2=r^2\\
\end{cases} \\
&\therefore \boldsymbol{c}\cdot \boldsymbol{b}=\left( -r-x \right) \cdot \left( r-x \right) +\left( -y \right) ^2=\left( x^2-r^2 \right) +y^2=0
\end{align*}

\begin{tcolorbox}
本题就是考察垂直的向量表示,也即内积为0的几何意义,较为简单。
\end{tcolorbox}

~

\begin{example}[综合运用15,难度:$\star $]
$\bigtriangleup ABC$的三边分别为$a,b,c$,边$BC$,$CA$,$AB$上的中线分别记为$m_a,m_b,m_c$,利用余弦定理证明
\begin{align*}
&m_a=\frac{1}{2}\sqrt{2\left( b^2+c^2 \right) -a^2} \\
&m_b=\frac{1}{2}\sqrt{2\left( a^2+c^2 \right) -b^2} \\
&m_c=\frac{1}{2}\sqrt{2\left( a^2+b^2 \right) -c^2}
\end{align*}
\end{example}

解一,用余弦定理证明:

\begin{figure}[h]
\centering
\begin{tikzpicture}[line join=round, scale=0.75]
\coordinate[label=above:{$A$}]   (A)  at (0.5,2);
\coordinate[label=left: {$B$}]   (B)  at (-2,0);
\coordinate[label=right:{$C$}]   (C)  at (2,0);
\coordinate[label=below:{$M$}]   (M)  at ($(B)!0.5!(C)$);
\coordinate[label=right:{$b$}]   (b)  at ($(A)!0.5!(C)$);
\coordinate[label=left: {$c$}]   (c)  at ($(A)!0.5!(B)$);
\coordinate[label=below:{$a/2$}] (a1) at ($(M)!0.5!(B)$);
\coordinate[label=below:{$a/2$}] (a2) at ($(M)!0.5!(C)$);
\coordinate[label=left: {$m_a$}] (ma) at ($(M)!0.5!(A)$);
\draw[thick] (A)--(B)--(C)--(A);
\draw[thick,blue] (A)--(M);
\pic["$\alpha $",draw,angle radius=0.3cm,angle eccentricity=1.5] {angle=A--M--B};
\pic["$\beta $",draw,angle radius=0.4cm,angle eccentricity=1.5] {angle=C--M--A};
\end{tikzpicture}
\end{figure}

如上图,从$\sin \left( \alpha +\beta \right) =\sin \alpha \cos \beta +\cos \alpha \sin \beta =0$开始。

分别使用余弦定理和正弦定理可得:
\[
\frac{c}{m_a}\sin B\cdot \frac{\frac{a^2}{4}+{m_a}^2-b^2}{2\cdot \frac{a}{2}\cdot m_a}+\frac{\frac{a^2}{4}+{m_a}^2-c^2}{2\cdot \frac{a}{2}\cdot m_a}\cdot \frac{b}{m_a}\sin C=0
\]
然后对$\sin B,\sin C$使用正弦定理可得:
\[
\frac{c}{m_a}\cdot \frac{\frac{a^2}{4}+{m_a}^2-b^2}{2\cdot \frac{a}{2}\cdot m_a}+\frac{\frac{a^2}{4}+{m_a}^2-c^2}{2\cdot \frac{a}{2}\cdot m_a}\cdot \frac{b}{m_a}\frac{c}{b}=0
\]
化简:
\[
\frac{a^2}{2}+2{m_a}^2-b^2-c^2=0
\]
略。

\begin{tcolorbox}
本题在于计算量非常大,先要找到合适的角,反反复复找角的过程计算量非常大。
\end{tcolorbox}

解二:

若不限制余弦定理,参考P39的例2,秒答,即证明:
\[
\left( 2m_a \right) ^2+a^2=2\left( b^2+c^2 \right)
\]

~

\begin{example}[综合运用17,难度:$\star \star $]
证明:设三角形的外接圆的半径是$R$,则$a=2R\sin A,b=2R\sin B,c=2R\sin C$。
\end{example}

解:

利用圆心角和圆周角的关系$\angle BOC=2\angle BAC$,加之$\cos 2A=1-\sin ^2A$,可得:
\begin{align*}
&\frac{R^2+R^2-a^2}{2RR}=1-2\sin ^2A \\
&1-\frac{a^2}{2R^2}=1-2\sin ^2A
\end{align*}
后略。

\begin{figure}[h]
\centering
\begin{tikzpicture}[line join=round, scale=1.25]
\draw[thick] (0,0) circle(1);
\coordinate                    (O) at (0,0);
\coordinate[label=above:{$A$}] (A) at (0.53,0.85);
\coordinate[label=left: {$B$}] (B) at (-0.72,-0.7);
\coordinate[label=right:{$C$}] (C) at (0.88,-0.47);
\draw[thick] (A)--(B)--(C)--(A);
\draw[thick,dashed,red] (B)--(O)--(C);
\pic["$2A$",draw,angle radius=0.3cm,angle eccentricity=1.5] {angle=B--O--C};
\end{tikzpicture}
\end{figure}

\begin{tcolorbox}
本题需要结合几何,不能纯靠向量知识,需要一定联想力。
\end{tcolorbox}

~

\begin{example}[拓广探索19,难度:$\star $]
如图,在平行四边形$ABCD$中,点$E,F$分别是$AD,DC$边的中点,$BE,BF$分别与$AC$交于$R,T$两点,你能发现$AR,RT,TC$之间的关系吗?用向量方法证明你的结论。
\end{example}

解:

初看似乎三条线段等长,最直观的方法就是建立如下坐标系,求$R$点的坐标。

\begin{figure}[h]
\centering
\begin{tikzpicture}[line join=round, scale=1.25]
\mydrawxy{-0.5}{3}{-0.5}{1.5}
\coordinate[label=below:      {$A$}]     (A) at (0,0);
\coordinate[label=below:      {$B\left( c,0 \right) $}]     (B) at (2,0);
\coordinate[label=above left: {$D\left( 2a,2b \right) $}]   (D) at (0.5,1);
\coordinate[label=above right:{$C\left( 2a+c,2b \right) $}] (C) at ($(B)+(D)$);
\coordinate[label=left:       {$E\left( a,b \right) $}]     (E) at ($(A)!0.5!(D)$);
\coordinate[label=above:      {$F$}]                        (F) at ($(D)!0.5!(C)$);
\draw[thick] (A)--(B)--(C)--(D)--(A);
\draw[thick,blue,name path=l1] (A)--(C);
\draw[thick,blue,name path=l2] (B)--(E);
\draw[thick,blue,name path=l3] (B)--(F);
\path [name intersections={of=l1 and l2}] coordinate[label=above:$R$] (R) at (intersection-1);
\path [name intersections={of=l1 and l3}] coordinate[label=above:$T$] (T) at (intersection-1);
\end{tikzpicture}
\end{figure}

易得$EB$和$AC$的直线方程:
\begin{align*}
&y-0=\frac{b}{a-c}\cdot \left( x-c \right) \\
&y=\frac{2b}{2a+c}\cdot x
\end{align*}
联立两方程求得$R$点坐标:
\[
R=\left( x,y \right) =\left( \frac{2a+c}{3},\frac{2b}{3} \right)
\]
确实有$AC=3AR$。

$TC$长度的讨论可以将$C$作为原点建立坐标系,略。

\begin{tcolorbox}
对于定量问题,放到坐标系下讨论,除了计算量大点,没啥缺点。
\end{tcolorbox}

~

\begin{example}[拓广探索20,难度:$\star $]
已知$\bigtriangleup ABC$的三个角$A,B,C$的对边分别为$a,b,c$,设$p=\frac{1}{2}\left( a+b+c \right) $,求证:

(1)三角形的面积$S=\sqrt{p\left( p-a \right) \left( p-b \right) \left( p-c \right)}$;

(2)若$r$为三角形的内切圆半径,则
\[
r=\sqrt{\frac{\left( p-a \right) \left( p-b \right) \left( p-c \right)}{p}}
\]

(3)把$BC,CA,AB$上的高分别记为$h_a,h_b,h_c$,则
\begin{align*}
&h_a=\frac{2}{a}\sqrt{p\left( p-a \right) \left( p-b \right) \left( p-c \right)} \\
&h_b=\frac{2}{b}\sqrt{p\left( p-a \right) \left( p-b \right) \left( p-c \right)} \\
&h_c=\frac{2}{c}\sqrt{p\left( p-a \right) \left( p-b \right) \left( p-c \right)}
\end{align*}
\end{example}

解:

(1)三角形面积
\begin{align*}
S&=\frac{1}{2}bc\sin A=\frac{1}{2}bc\sqrt{1-\cos ^2A} \\
&=\frac{1}{2}bc\sqrt{1-\left( \frac{b^2+c^2-a^2}{2bc} \right) ^2} \\
&=\frac{1}{4}\sqrt{a^4-b^4-c^4+2a^2b^2+2a^2c^2+2b^2c^2}
\end{align*}
而$\sqrt{p\left( p-a \right) \left( p-b \right) \left( p-c \right)}$展开后相等,证毕。

(2)以内切圆圆心为基点,将三角形分成3部分,用下面的思路解,略
\[
S_{\bigtriangleup ABC}=S_{\bigtriangleup ABO}+S_{\bigtriangleup AOC}+S_{\bigtriangleup OBC}
\]

\begin{tcolorbox}
本题计算量略大,方法还是很直观的。
\end{tcolorbox}

~

\begin{example}[拓广探索23,难度:$\star \star $]
已知$a,b,c$分别为$\bigtriangleup ABC$三个内角$A,B,C$的对边,且
\[
a\cos C+\sqrt{3}a\sin C-b-c=0
\]
\begin{enumerate}
    \item 求$A$;
    \item 若$a=2$,且$\bigtriangleup ABC$的面积为$\sqrt{3}$,求$b,c$。
\end{enumerate}
\end{example}

\begin{figure}[h]
\centering
\begin{tikzpicture}[line join=round, scale=0.75]
\coordinate[label=above:     {$A$}]  (A)  at (0.5,2);
\coordinate[label=left:      {$B$}]  (B)  at (-2,0);
\coordinate[label=right:     {$C$}]  (C)  at (2,0);
\coordinate[label=above left:{$c$}]  (c)  at ($(B)!0.5!(A)$);
\coordinate[label=left:      {$b$}]  (b)  at ($(C)!0.5!(A)$);
\coordinate[label=below:     {$a$}]  (a)  at ($(B)!0.5!(C)$);
\coordinate[label=right:     {$A'$}] (A') at ($(B)!1.8!(A)$);
\coordinate[label=above left:{$b$}]  (b') at ($(A)!0.5!(A')$);
\draw[thick] (A)--(B)--(C)--(A);
\draw[thick,dashed,red] (A)--(A')--(C);
\pic["$\pi -A$",draw,angle radius=0.4cm,angle eccentricity=2.5] {angle=C--A--A'};
\pic["$A/2$",draw,angle radius=0.4cm,angle eccentricity=1.5] {angle=A--A'--C};
\pic["$A/2$",draw,angle radius=0.4cm,angle eccentricity=1.5] {angle=A'--C--A};
\end{tikzpicture}
\end{figure}

解:

(1)分析已知等式:
\begin{align*}
&2a\left( \frac{1}{2}\cos C+\frac{\sqrt{3}}{2}\sin C \right) =b+c \\
&2a\sin \left( C+\frac{\pi}{6} \right) =b+c
\end{align*}
出现$\frac{b+c}{a}$,于是构建如上三角形,对$\bigtriangleup A'BC$用正弦定理,并结合已知等式可得:
\begin{align*}
&\because \frac{c+b}{\sin \left( C+\frac{A}{2} \right)}=\frac{a}{\sin \frac{A}{2}} \\
&\therefore \sin \left( C+\frac{A}{2} \right) =\frac{c+b}{a}\sin \frac{A}{2} \\
&\therefore \sin C\cos \frac{A}{2}+\cos C\sin \frac{A}{2}=\frac{c+b}{a}\sin \frac{A}{2} \\
&\therefore \left( a\cot \frac{A}{2} \right) \sin C+a\cos C=b+c \\
&\therefore \cot \frac{A}{2}=\sqrt{3} \\
&\therefore A=\frac{\pi}{3}
\end{align*}

(2)有了$A$,结合三角形公式和余弦定理可得方程组:
\begin{align*}
&\because S=\frac{1}{2}\cdot bc\cdot \sin A=\sqrt{3} \\
&\because \cos A=\frac{b^2+c^2-a^2}{2bc} \\
&\therefore b=c=2
\end{align*}

解二:

从已知等式直接入手:
\begin{align*}
&\because a\cos C+\sqrt{3}a\sin C-b-c=0 \\
&\therefore a\frac{a^2+b^2-c^2}{2ab}+\sqrt{3}c\sin A=b+c \\
&\therefore \sin A=\frac{2b\left( b+c \right) -\left( a^2+b^2-c^2 \right)}{2\sqrt{3}bc}=\frac{1}{\sqrt{3}}\cdot \left( \cos A+1 \right) \\
&\because \sin ^2A+\cos ^2A=1 \\
&\therefore A=\frac{\pi}{3}
\end{align*}

\begin{tcolorbox}
此题有一定难度,解一需要细心观察,解二计算量大,没啥说的。
\end{tcolorbox}




