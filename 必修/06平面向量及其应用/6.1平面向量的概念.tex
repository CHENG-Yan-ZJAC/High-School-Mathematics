\section{平面向量的概念}

都是定义,没啥难点。向量就是带方向的量。

可以这么理解,如果描述一维空间的坐标,则使用标量即可,没有方向,或者只有两个方向,前后,用实数的正负即可。如果描述二维空间的坐标,则必须使用两个标量,组合在一起就是向量。

教材中说道向量的三元素:起点、方向、长度。其实只有方向和长度,我们不太关心起点,起点是可以移动的。从后续的平行和相等的定义也可以看出,起点并没有要求。

这里容易忽略一个关键点,即模和方向是向量的固有属性。所谓的固定属性就是它们不随参考系的变化而变化,这点在学习到第3节的时候需要特别注意!XML。




