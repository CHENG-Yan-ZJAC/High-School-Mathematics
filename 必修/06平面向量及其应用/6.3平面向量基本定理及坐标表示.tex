\section{平面向量基本定理及坐标表示}

本节要点:
\begin{itemize}
    \item 理解向量的坐标表示;
    \item 熟练掌握向量运算的坐标表示。
\end{itemize}

\begin{tcolorbox}
本节将向量分解为实数的有序组合,并定义运算法则。
\end{tcolorbox}

本节的概念较为复杂,新概念多,新写法多,再次作梳理总结。

首先,高中阶段讨论的向量是二维向量,几何上表示二维平面的有向线段,在{\it xy}坐标系中,原点和任意点也能构成的有向线段。加之“平面向量定理”,我们可以证明二维平面和{\it xy}坐标系是同构的。所以,以下向量的表示方法是等价的:
\begin{itemize}
    \item $\overrightarrow{AB}$:表示二维空间的有向线段;
    \item $x\boldsymbol{i}+y\boldsymbol{j}$:表示{\it xy}坐标系中的有向线段;
    \item $\left( x,y \right) $:表示{\it xy}坐标系中的点。
\end{itemize}
综合表示如下,注意,这里的=表示含义等价,而非代数中的等量:
\[
\boldsymbol{a}=\overrightarrow{AB}=x\boldsymbol{i}+y\boldsymbol{j}=\left( x,y \right)
\]
仔细研读本节所有例题,理解上述关系式。

于是我们不难得到:
\begin{align*}
&\boldsymbol{a}+\boldsymbol{b}=\overrightarrow{AB}=\left( x_{\boldsymbol{a}}+x_{\boldsymbol{b}} \right) \boldsymbol{i}+\left( y_{\boldsymbol{a}}+y_{\boldsymbol{b}} \right) \boldsymbol{j}=\left( x_{\boldsymbol{a}}+x_{\boldsymbol{b}},y_{\boldsymbol{a}}+y_{\boldsymbol{b}} \right) \\
&\lambda \boldsymbol{a}=\lambda x_{\boldsymbol{a}}\boldsymbol{i}+\lambda y_{\boldsymbol{a}}\boldsymbol{j}=\left( \lambda x_{\boldsymbol{a}},\lambda y_{\boldsymbol{a}} \right) \\
&\boldsymbol{a}\cdot \boldsymbol{b}=\left( x_{\boldsymbol{a}}\boldsymbol{i}+y_{\boldsymbol{a}}\boldsymbol{j} \right) \cdot \left( x_{\boldsymbol{b}}\boldsymbol{i}+y_{\boldsymbol{b}}\boldsymbol{j} \right) =x_{\boldsymbol{a}}\cdot x_{\boldsymbol{b}}+y_{\boldsymbol{a}}\cdot y_{\boldsymbol{b}}=\left| \boldsymbol{a} \right|\left| \boldsymbol{b} \right|\cos \alpha
\end{align*}

\begin{tcolorbox}
本节将向量用一个有序实数对表示,并根据之前的定义完善了向量的运算,使得向量彻底关联了代数和几何。
\end{tcolorbox}

\begin{figure}[h]
\centering
\begin{minipage}{.49\textwidth}
\centering
\begin{tikzpicture}[line join=round, scale=1]
\coordinate (x1) at (-1,0);
\coordinate (y1) at (-0.5,-1);
\coordinate[label=right:{$x$}] (x2) at (1,0);
\coordinate[label=above:{$y$}] (y2) at (0.5,1);
\draw[->,red] (x1)--(x2);
\draw[->,red] (y1)--(y2);
\end{tikzpicture}
\end{minipage}
\begin{minipage}{.49\textwidth}
\centering
\begin{tikzpicture}[line join=round, scale=1]
\mydrawxy{-1}{1}{-1}{1}
\end{tikzpicture}
\end{minipage}
\end{figure}

还有一个问题需要注意,教材中未提及。坐标系可以多种多样,只要两个坐标轴不重合即可构建二维坐标系,如上左图,若约束{\it xy}垂直,就构成正交坐标系,也即我们熟知的笛卡尔坐标系,如上右图。

之前在6.1中我提及,模是向量的固有属性,所以要计算$\boldsymbol{a}=\left( x,y \right) $的模,这里的$x,y$必须是正交坐标系下的坐标!高中教材不对模下一个明确的定义,这部分在《线性代数》中定义,所以这里只要知道模的计算需要放在直角坐标系下即可。

\begin{tcolorbox}
高中阶段的向量是简化版的线性代数,XML。
\end{tcolorbox}

~

\begin{example}[复习巩固9,难度:$\star $]
已知$\left| \boldsymbol{a} \right|=3,\boldsymbol{b}=\left( 1,2 \right) $,且$\boldsymbol{a}\parallel \boldsymbol{b}$,求$\boldsymbol{a}$的坐标。
\end{example}

解:

令$\boldsymbol{a}=\left( x,y \right) $,则有:
\begin{align*}
&x^2+y^2=9 \\
&\left( x,y \right) =\lambda \left( 1,2 \right)
\end{align*}
第2个式子表示$\left( x,y \right) =\lambda \left( 1,2 \right) $,不难解得:
\[
\boldsymbol{a}=\pm \left( \frac{3}{\sqrt{5}},\frac{6}{\sqrt{5}} \right)
\]

\begin{tcolorbox}
本题考察向量的性质,并不难。
\end{tcolorbox}

~

\begin{example}[复习巩固10,难度:$\star $]
已知$\boldsymbol{a}=\left( 4,2 \right) $,求与$\boldsymbol{a}$垂直的单位向量的坐标。
\end{example}

解:

与一个向量垂直,约束了其方向,单位向量,约束了其大小,不难发现该题必然得到两个方程:
\begin{align*}
&\boldsymbol{a}\cdot \boldsymbol{b}=0 \\
&\left| \boldsymbol{b} \right|=1
\end{align*}
令$\boldsymbol{b}=\left( x,y \right) $可求解,略。

\begin{tcolorbox}
本题依然考察向量的性质,并不难。
\end{tcolorbox}

~

\begin{example}[综合运用14,难度:$\star $]
求证:以$A\left( 1,0 \right) $,$B\left( 5,-2 \right) $,$C\left( 8,4 \right) $,$D\left( 4,6 \right) $为顶点的四边形是一个矩形。
\end{example}

\begin{figure}[h]
\centering
\begin{tikzpicture}[line join=round, scale=0.25]
\mydrawxy{-3}{10}{-3}{8}
\coordinate[label=left: {$A$}]              (A) at (1,0);
\coordinate[label=right:{$B$}]              (B) at (5,-2);
\coordinate[label=right:{$C$}]              (C) at (8,4);
\coordinate[label=left: {$D$}]              (D) at (4,6);
\coordinate[label=below:{$\boldsymbol{a}$}] (a) at ($(A)!0.5!(B)$);
\coordinate[label=right:{$\boldsymbol{b}$}] (b) at ($(B)!0.5!(C)$);
\coordinate[label=above:{$\boldsymbol{c}$}] (c) at ($(C)!0.5!(D)$);
\coordinate[label=left: {$\boldsymbol{d}$}] (d) at ($(D)!0.5!(A)$);
\draw[thick,-stealth] (A)--(B);
\draw[thick,-stealth] (B)--(C);
\draw[thick,-stealth] (C)--(D);
\draw[thick,-stealth] (D)--(A);
\end{tikzpicture}
\end{figure}

解:

从矩形的定义出发,有一个角是直角的平行四边形。
可令向量$\boldsymbol{a},\boldsymbol{b},\boldsymbol{c},\boldsymbol{d}$如上图,不难发现,只需证明:
\begin{align*}
&\boldsymbol{a}=\lambda \boldsymbol{c} \\
&\boldsymbol{a}\cdot \boldsymbol{b}=0
\end{align*}
略。

\begin{tcolorbox}
本题考察向量运算的几何意义,并不难。
\end{tcolorbox}

~

\begin{example}[拓广探索15,难度:$\star $]
如图,$Ox,Oy$是平面内相交成60°角的两条数轴,$\boldsymbol{e}_1,\boldsymbol{e}_2$分别是与{\it x}轴、{\it y}轴正方向通向的单位向量。若向量$\overrightarrow{OP}=x\boldsymbol{e}_1+y\boldsymbol{e}_2$,则把有序数对$\left( x,y \right) $叫做向量$\overrightarrow{OP}$在坐标系$Oxy$中的坐标。设$\overrightarrow{OP}=3\boldsymbol{e}_1+2\boldsymbol{e}_2$,
\begin{enumerate}
    \item 计算$\left| \overrightarrow{OP} \right|$;
    \item 根据平面向量基本定理判断,本题中对向量坐标的规定是否合理。
\end{enumerate}
\end{example}

\begin{figure}[h]
\centering
\begin{tikzpicture}[line join=round, scale=0.75]
\coordinate[label=below left:{$O$}]           (O)  at (0,0);
\coordinate[label=below:{$\boldsymbol{e}_1$}] (E1) at (1,0);
\coordinate[label=left: {$\boldsymbol{e}_2$}] (E2) at (0.5,0.866);
\coordinate[label=right:{$x$}]                (x)  at ($(O)!3.5!(E1)$);
\coordinate[label=above:{$y$}]                (y)  at ($(O)!2.5!(E2)$);
\draw[->,red] (O)--(x);
\draw[->,red] (O)--(y);
\draw[thick,-stealth] (O)--(E1);
\draw[thick,-stealth] (O)--(E2);
\coordinate                    (Px) at ($(O)!3!(E1)$);
\coordinate                    (Py) at ($(O)!2!(E2)$);
\coordinate[label=right:{$P$}] (P)  at ($(Px)+(Py)$);
\draw[thick,-stealth] (O)--(P);
\draw[dashed] (Py)--(P)--(Px);
\end{tikzpicture}
\end{figure}

解:

(1)当前坐标系不是正交坐标系,所以$\left| \overrightarrow{OP} \right|\ne 3^2+2^2$,而是需要转换到正交坐标系。
\begin{align*}
&\because \begin{cases}
	\boldsymbol{e}_1=1\cdot \boldsymbol{i}+0\cdot \boldsymbol{j}\\
	\boldsymbol{e}_2=\frac{1}{2}\cdot \boldsymbol{i}+\frac{\sqrt{3}}{2}\cdot \boldsymbol{j}\\
\end{cases} \\
&\therefore \overrightarrow{OP}=3\boldsymbol{e}_1+2\boldsymbol{e}_2=3\left( 1\cdot \boldsymbol{i}+0\cdot \boldsymbol{j} \right) +2\left( \frac{1}{2}\cdot \boldsymbol{i}+\frac{\sqrt{3}}{2}\cdot \boldsymbol{j} \right) =4\boldsymbol{i}+\sqrt{3}\boldsymbol{j} \\
&\therefore \left| \overrightarrow{OP} \right|=\sqrt{4^2+\sqrt{3}^2}
\end{align*}

(2)合理,不展开,XML。

\begin{tcolorbox}
本题考察向量的模的定义,由于教材缺乏明确定义,所以会有些迷惑。
\end{tcolorbox}

~

\begin{example}[拓广探索16,难度:$\star \star $]
用向量方法证明:对于任意的$a,b,c,d\in \mathbb{R} $,恒有不等式
\[
\left( ac+bd \right) ^2\leqslant \left( a^2+b^2 \right) \left( c^2+d^2 \right)
\]
\end{example}

解:

令$\boldsymbol{a}=\left( a,b \right) ,\boldsymbol{b}=\left( c,d \right) $则,
\begin{align*}
&\left( a^2+b^2 \right) \left( c^2+d^2 \right) =\left| \boldsymbol{a} \right|^2\cdot \left| \boldsymbol{b} \right|^2 \\
&\left( \boldsymbol{a}\cdot \boldsymbol{b} \right) ^2=\left( ac+bd \right) ^2
\end{align*}
于是:
\begin{align*}
&\because \cos \alpha =\frac{\boldsymbol{a}\cdot \boldsymbol{b}}{\left| \boldsymbol{a} \right|\cdot \left| \boldsymbol{b} \right|}\in \left[ -1,1 \right] \\
&\therefore \frac{\left( \boldsymbol{a}\cdot \boldsymbol{b} \right) ^2}{\left| \boldsymbol{a} \right|^2\cdot \left| \boldsymbol{b} \right|^2}\in \left[ 0,1 \right]
\end{align*}
当且仅当$\boldsymbol{a},\boldsymbol{b}$平行时等号成立。

\begin{tcolorbox}
本题需要一些想象力,如果没有提示用向量方法,还是有些难度的。
\end{tcolorbox}

\begin{tcolorbox}
这里引申出一个话题。我们在规定向量的坐标$\boldsymbol{a}=\left( x,y \right) $时,并没有对两个坐标量有所约束,更没有要求坐标系必须是直角坐标。加之向量运算的底层规则还是建立在实数的运算法则上,所以本题可以用向量的方法。
\end{tcolorbox}




