\section{本章小结}

本章介绍了平面上的向量,重点:
\begin{itemize}
    \item 向量的运算及其几何意义;
    \item 余弦定理和正弦定理。
\end{itemize}

向量是一个全新的域,还有下一章的复数,和一贯以来的实数不太一样,所以仅用一章的课时难以适应,需要反复阅读概念、推导定理和性质,并结合几何深入思考。向量是代数和几何的桥梁,所以容易出大题和难题,更需要多读多练。

~

\begin{example}[综合运用15,难度:$\star $]
已知$\bigtriangleup P_1P_2P_3$,向量$\overrightarrow{OP_1},\overrightarrow{OP_2},\overrightarrow{OP_3}$满足条件$\overrightarrow{OP_1}+\overrightarrow{OP_2}+\overrightarrow{OP_3}=\mathbf{0},\left| \overrightarrow{OP_1} \right|=\left| \overrightarrow{OP_2} \right|=\left| \overrightarrow{OP_3} \right|$,求证:$\bigtriangleup P_1P_2P_3$是等边三角形。
\end{example}

解:

即通过已知条件求证$\left| \overrightarrow{P_1P_2} \right|=\left| \overrightarrow{P_2P_3} \right|=\left| \overrightarrow{P_3P_1} \right|$。使用{\it xy}坐标系,两个已知条件可表示为:
\begin{align*}
&\because \overrightarrow{OP_1}+\overrightarrow{OP_2}+\overrightarrow{OP_3}=\mathbf{0} \\
&\therefore \begin{cases}
	\left( x_{P1}-x_O \right) +\left( x_{P2}-x_O \right) +\left( x_{P3}-x_O \right) =0\\
	\left( y_{P1}-y_O \right) +\left( y_{P2}-y_O \right) +\left( y_{P3}-y_O \right) =0\\
\end{cases} \\
&\therefore \begin{cases}
	x_{P1}+x_{P2}+x_{P3}=3x_O\\
	y_{P1}+y_{P2}+y_{P3}=3x_O\\
\end{cases} \\
&\because \left| \overrightarrow{OP_1} \right|=\left| \overrightarrow{OP_2} \right|=\left| \overrightarrow{OP_3} \right| \\
&\therefore \left( x_{P1}-x_O \right) ^2+\left( y_{P1}-y_O \right) ^2=\left( x_{P2}-x_O \right) ^2+\left( y_{P2}-y_O \right) ^2 \\
&=\left( x_{P3}-x_O \right) ^2+\left( y_{P3}-y_O \right) ^2
\end{align*}
将$\left( x_{P1}-x_O \right) ^2+\left( y_{P1}-y_O \right) ^2=\left( x_{P2}-x_O \right) ^2+\left( y_{P2}-y_O \right) ^2$部分展开,并将$x_O,y_O$替换掉可得:
\[
\left( x_{P1}-x_{P3} \right) ^2+\left( y_{P1}-y_{P3} \right) ^2=\left( x_{P2}-x_{P3} \right) ^2+\left( y_{P2}-y_{P3} \right) ^2
\]
余下略。

\begin{tcolorbox}
本题思路还是明确的,计算量偏大而已。
\end{tcolorbox}

~

\begin{example}[综合运用16,难度:$\star $]
如图,已知$\overrightarrow{OA}=\boldsymbol{a},\overrightarrow{OB}=\boldsymbol{b}$,任意点$M$关于点$A$的对称点为$S$,点$S$关于点$B$的对称点为$N$,用$\boldsymbol{a},\boldsymbol{b}$表示向量$\overrightarrow{MN}$。(本题可以运用信息技术发现规律)
\end{example}

\begin{figure}[h]
\centering
\begin{tikzpicture}[line join=round, scale=0.75]
\coordinate[label=left:       {$O$}]              (O) at (0,0);
\coordinate[label=below:      {$A$}]              (A) at (1.5,1);
\coordinate[label=above right:{$B$}]              (B) at (2,2);
\coordinate[label=left:       {$M$}]              (M) at (0.1,1.5);
\coordinate[label=right:      {$S$}]              (S) at ($(M)!2!(A)$);
\coordinate[label=above:      {$N$}]              (N) at ($(S)!2!(B)$);
\coordinate[label=below:      {$\boldsymbol{a}$}] (a) at ($(O)!0.5!(A)$);
\coordinate[label=left:       {$\boldsymbol{b}$}] (b) at ($(O)!0.75!(B)$);
\draw[thick] (M)--(S)--(N);
\draw[thick,-stealth] (M)--(N);
\draw[thick,-stealth] (O)--(A);
\draw[thick,-stealth] (O)--(B);
\draw[dashed] (A)--(B);
\end{tikzpicture}
\end{figure}

解一:

令$A=\left( x_A,y_A \right) ,B=\left( x_B,y_B \right) ,M=\left( x_M,y_M \right) $,则
\[
\overrightarrow{MN}=\left( x_N-x_M,y_N-y_M \right)
\]
对已知条件整理:
\begin{align*}
&\because \overrightarrow{MS}=2\overrightarrow{AS} \\
&\therefore \begin{cases}
	x_S-x_M=2\left( x_S-x_A \right)\\
	y_S-y_M=2\left( y_S-y_A \right)\\
\end{cases} \\
&\because \overrightarrow{SN}=2\overrightarrow{SB} \\
&\therefore \begin{cases}
	x_N-x_S=2\left( x_B-x_S \right)\\
	y_N-y_S=2\left( y_B-y_S \right)\\
\end{cases} \\
&\therefore \begin{cases}
	x_N-x_M=2\left( x_B-x_A \right)\\
	y_N-y_M=2\left( y_B-y_A \right)\\
\end{cases} \\
&\therefore \overrightarrow{MN}=2\left( x_B-x_A,y_B-y_A \right) =2\left( \boldsymbol{b}-\boldsymbol{a} \right)
\end{align*}

解二:

添加辅助线$AB$,从三角形不难得到$\overrightarrow{MN}=2\overrightarrow{AB}$,余下略。

\begin{tcolorbox}
解一纯用坐标系,思路简单,计算量大。解二结合几何,计算量小。
\end{tcolorbox}




