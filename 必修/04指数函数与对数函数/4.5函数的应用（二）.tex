\section{函数的应用(二)}

本节没有难度,需要注意几点:
\begin{itemize}
    \item “4.5.1函数的零点与方程的解”中,零点存在定理需要严格定义“连续不断”这个概念,这点在高等数学微积分中展开,高中阶段只是了解该定理;
    \item “4.5.3函数模型的应用”中,马尔萨斯模型的问题在于增长率为常数这个假设并不正确。
\end{itemize}

~

函数模型的应用中,我们首先进行数据的采集,根据数据绘制图形,发掘数据的关系,然后选择合适的模型,可以是幂函数、指数函数、对数函数或者它们的组合,最后根据数据决定未知数,这需要线性代数中的最小二乘法及其衍生方法。当然,还需要对模型进行评判,这需要数理统计中的检验假设的相关知识。高中阶段我们考虑地较为简单,使用单个的标准函数作为模型即可。

\begin{tcolorbox}
还是那个套路,只不过我们原来只有一个框“幂函数”,现在我们多了一个“指数函数”这个框。XML!
\end{tcolorbox}




