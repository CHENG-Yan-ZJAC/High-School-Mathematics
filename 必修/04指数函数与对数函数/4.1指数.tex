\section{指数}

略

~

\begin{example}[拓广探索9,难度:$\star $]
从盛有1L纯酒精的容器中倒出1/3L,然后用水填满;再倒出1/3L,又用水填满……
\begin{enumerate}
    \item 连续进行5次,容器中的纯酒精还剩下多少?
    \item 连续进行n次,容器中的纯酒精还剩下多少?
\end{enumerate}
\end{example}

解:

令酒精浓度$\rho $为酒精容量比容器容量。

初始时,酒精浓度$\rho _0=1$,第一次倒酒填水后,酒精浓度为$\rho _1=\frac{2}{3}$。假设第$n$次倒酒填水完毕后酒精浓度为$\rho _n$,则第$n+1$次倒出1/3L后,容器内酒精容量为$\rho _n\cdot \frac{2}{3}$,用水填满后,酒精浓度为
\[
\rho _{n+1}=\frac{\rho _n\cdot \frac{2}{3}}{1}=\rho _n\cdot \frac{2}{3}
\]
于是我们可以得到,容器内酒精在第$n$次倒酒填水完毕后的浓度为:
\[
\rho _n=\left( \frac{2}{3} \right) ^n
\]
为一个首相为$\frac{2}{3}$,公比为$\frac{1}{3}$的等比数列。

余下略。

\begin{tcolorbox}
本题就是数列,找到通项公式,余下的体力活。本题是一个很好的引子,指数函数其实就是等比数列的稠密化,下节展开。
\end{tcolorbox}




