\section{函数的概念及其表示}

本节要点:
\begin{itemize}
    \item 深刻掌握函数的概念。
\end{itemize}

~

教材从无开始介绍函数,有了基础知识,我们可以给函数做一个精确的定义。

\begin{definition}[映射]
设非空集合$X,Y$,若对于某种确定的法则$f$,对于$\forall x\in X$,$Y$都有唯一元素$y$与之对应,则称$f$为{\bf 从集合$X$到集合$Y$的映射},记作:
\begin{align*}
&f:X\mapsto Y \\
&f:x\mapsto y=f\left( x \right) ,x\in X
\end{align*}
若有$\varphi :X\mapsto U_1,f:U_2\mapsto Y$且$U_1\subseteq U_2$,则从$X$到$Y$存在唯一确定的法则,使得$\forall x\in X$,$Y$都有唯一元素$y$与之对应,我们称$X$到$Y$的这种映射为{\bf 复合映射},也可称为{\bf 映射的乘积},记作:
\begin{align*}
&f\circ \varphi :X\mapsto Y \\
&f\circ \varphi :x\mapsto y=f\left[ \varphi \left( x \right) \right] ,x\in X
\end{align*}
若对于映射$f:X\mapsto Y$,$\forall y\in Y$,在$X$中都有唯一的原像与之对应,则称从$Y$到$X$的这种映射为{\bf 逆映射},记作:
\begin{align*}
&f^{-1}:Y\mapsto X \\
&f^{-1}:y\mapsto x=f^{-1}\left( y \right) ,y\in Y
\end{align*}
\end{definition}

根据集合的不同,映射在数学中也称“函数”、“算子”、“变换”等。高中数学中我们讨论实数构成的集合,所以称为函数;线性代数中我们讨论向量组成的集合,常称为变换。

\begin{definition}[函数]
设$X,Y$为两个非空实数集,$f$为$X\mapsto Y$的一个映射,则称$f$为{\bf 定义在$X$上的函数(function)},记作
\[
y=f\left( x \right) \quad x\in X
\]
其中:
\begin{itemize}
    \item $f$:{\bf 映射关系};
    \item $x,y$:{\bf 自变量(independent variable)},{\bf 因变量(dependent variable)};
    \item $X,Y$:{\bf 定义域(domain)},{\bf 值域(range)}。
\end{itemize}
有些函数,其因变量可以明显地表达成自变量的解析式$y=f\left( x \right) $,我们称为{\bf 显函数},有些则无法用解析式明显地表达,但可以确定$y$是$x$的函数,我们用
\[
F\left( x,y \right) =0
\]
表示,称为{\bf 隐函数(implicit function)}。
\end{definition}

函数最重要的是“映射关系$f$”和“定义域$X$”,只要这两个一样,就说是同一个函数,至于自变量、因变量、映射关系具体用哪个符号,都不重要。

高中阶段我们讨论的函数都只有一个自变量,当然自变量的个数不限于一个,可以有两个或更多,称为多元函数,进一步可细分为多元数量值函数和多元向量值函数,也是有实际意义的。如我们在描述地貌的高度时,自变量可以是经度和纬度,因变量则是高度,即:
\[
z=h\left( x,y \right)
\]

函数可以通过数学表达式表示,如$y=ax^2$,也可以通过图形表示,通常用笛卡尔坐标系中的点表示,若$x,y$的取值为实数,则这些点是稠密的,将连成一条直线或曲线。但无论如何,数学表达式和图形是一一对应的,数学表达式的优点是清晰和可定量化分析,图形的优点是直观、易于从总体上把握函数的形状和性质,学习的时候需要特别注意数形结合。

\begin{tcolorbox}
初学函数这个概念,特别容易跟“公式”混淆。确实,高中的函数跟公式是一个意思,两者是一个概念,都表示两个变量之间的准确的量化关系。区别在于使用场景和侧重点,如下例,正方形面积公式和自由落体公式:
\[
S=a^2 \qquad h=\frac{1}{2}gt^2
\]
公式突出的是两个变量的物理意义,正方形面积公式表达的是边长和面积的关系,自由落体公式表达的是下落高度和时间的关系,这两个式子在物理上有不同的涵义。而在数学上看来,这两个公式是一样的,都可以归类为:
\[
y=ax^2
\]
只是系数$a$有差别罢了。数学就是将这类物理上不同,但数学上相同的概念抽象出来,加以研究。如上$y=ax^2$,数学关心的是函数的有无最值、变化趋势如何、有无周期性、是增是减等等。
\end{tcolorbox}

\begin{tcolorbox}
于是,我们学习函数的目标就非常明显了。我们需要学习这么几种函数:幂函数、指数函数、对数函数、三角函数、反三角函数。重点学习它们的定义域、值域、变化趋势。实际应用中,首先根据题意建模,即用函数的语言描述问题,得到的函数必然会落到上述的5种函数中,然后根据我们学习到的函数的性质进行分析,回答诸如成本最小、时间最短、几何上两点距离最短等问题。结合第一章中的数学范式思考体会!XML。
\end{tcolorbox}




