\section{函数的基本性质}

本节要点:
\begin{itemize}
    \item 掌握函数的各种性质。
\end{itemize}

~

关于函数的性质有最值(更准确的说是有界性)、单调性、奇偶性、周期性等概念,具体见教材不再赘述。

在分析函数的性质时,还是要注意从定义出发,如判断某个函数的单调性:
\begin{enumerate}
    \item 任取两个点,设$x_1<x_2$;
    \item 带入函数,用前一章的不等式判断$y_1,y_2$的关系;
    \item 根据定义得出单调性。
\end{enumerate}

\begin{tcolorbox}
函数的性质依然逃不开第一章的思维方法,即我们从函数的定义出发总结各种函数的性质,或从性质的定义出发判断函数的性质,将这种性质运用到对实际问题的判断中。如判断如何用相等的材料制造最大容量的水桶,其实就是判断函数的最大值。
\end{tcolorbox}

~

\begin{example}[综合运用8(1),难度:$\star $]
根据函数单调性的定义证明函数$y=x+\frac{9}{x}$在区间$\left[ 3,+\infty \right) $上单调递增。
\end{example}

解:

这类问题还是要从定义出发。
令$x_1<x_2\in \left[ 3,+\infty \right) $,则:
\begin{align*}
\left( x_1+\frac{9}{x_1} \right) -\left( x_2+\frac{9}{x_2} \right) &=x_1+\frac{9}{x_1}-x_2-\frac{9}{x_2} \\
&=\frac{\left( x_1-x_2 \right) x_1x_2-9\left( x_1-x_2 \right)}{x_1x_2} \\
&=\frac{\left( x_1-x_2 \right) \left( x_1x_2-9 \right)}{x_1x_2}
\end{align*}
由于在区间$\left[ 3,+\infty \right) $上必有$x_1x_2>9$,故单调递增,证毕。

\begin{tcolorbox}
不少证明题是直接从定义出发的,不需要定理的参与,所以不可忽视对定义的掌握和理解。
\end{tcolorbox}




